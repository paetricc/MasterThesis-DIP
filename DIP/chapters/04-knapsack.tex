\chapter{Optimalizace úlohy batohu pomocí kvantově inspirovaných evolučních algoritmů}

Problém batohu patří mezi základní kombinatorické optimalizační úlohy, jenž nachází široké uplatnění v oblastech, jako jsou logistika, ekonomie nebo plánování zdrojů a patří do skupiny NP těžkých problémů. 
Jednou z možností, jak dosáhnout dostatečně kvalitního řešení v~rozumném čase, je využití kvantově inspirovaných evolučních algoritmů, které využívají principy kvantové mechaniky a biologické evoluce. 

Kapitola nejdříve poskytne úvod do konceptu řešeného problému batohu, následně popíše reprezentaci tohoto problému v kontextu kvantově inspirovaných evolučních algoritmů a v poslední části se bude věnovat návrhu jednotlivých experimentů, přičemž jejich výsledky budou vyhodnoceny v kapitole~\ref{chapt:experiments}.

\section{Základní popis problému batohu}
Problém batohu (\emph{knapsack problem}) patří do třídy NP těžkých problémů, což znamená, že neexistuje žádný algoritmus, který by ho dokázal řešit v polynomiálním čase, ale zároveň není dokázáno, že takový algoritmus neexistuje.
Problém batohu je kombinatorický optimalizační problém, jehož cílem je maximalizovat celkovou hodnotu položek v batohu, přičemž každá z~položek má svoji určitou hodnotu a váhu. 
Samotná kapacita batohu je omezena na určitou hodnotu a tudíž součet vah jednotlivých položek v~batohu nesmí překročit jeho kapacitu. 
Pro každou z položek je možné rozhodnout zda bude do batohu vložena či nikoli, přičemž každá z položek může být v batohu právě jednou~\cite{ApproximationAlgorithms,IntroductionToAlgorithms}. 

Formálněji řečeno, mějme množinu $n$ položek, přičemž každá položka $i$ je popsána dvojicí $\left( v_i, w_i \right)$, kde
\begin{itemize}
    \item $v_i$ reprezentuje hodnotu položky $i$ a
    \item $w_i$ značí její váhu.
\end{itemize}
Hodnota rozhodovací proměnné $x_i$ je určena následovně:
\begin{equation*}
    x_i =
    \begin{cases}
        1 & \text{pokud je položka } i \text{ umístěna do batohu}, \\
        0 & \text{jinak}.
    \end{cases}    
\end{equation*}
Hlavním cílem je maximalizovat celkovou hodnotu položek v batohu:
\begin{equation*}
    \max \sum_{i=1}^{n} v_i x_i,
\end{equation*}
přičemž součet vah položek nesmí překročit celkovou kapacitu $C$ batohu:
\begin{equation} \label{eq:knapsack-capacity}
    \sum_{i=1}^{n} w_i x_i \leq C,
\end{equation}
kde:
\begin{equation*}
    x_i \in \{0,1\} \text{ pro } \forall i = 1,2,\dots,n.
\end{equation*}

Varianta problém batohu popsaná výše je známá jako \emph{0-1 knapsack problem}~\cite{knapsack01-survey}. 
Další možné varianty problému batohu jsou:
\begin{itemize}
    \item \textbf{Vícerozměrný problém batohu (\emph{multi-dimensional knapsack problem}):} Jed\-ná se o zobecněný 0-1 problém batohu, kde má batoh více různých omezení~\cite{multidimensional-knapsack}.
    \item \textbf{Zlomkový problém batohu (\emph{fractional knapsack problem}):} Jednotlivé položky lze dělit na části, tudíž je možné do batohu umístit i pouze část položky~\cite{IntroductionToAlgorithms}.
    \item \textbf{Neomezený problém batohu (\emph{unbounded knapsack problem}):} Každá položka může být do batohu umístěna v neomezeném počtu~\cite{unbounded-knapsack}.
    \item \textbf{Vícevýběrový problém batohu (\emph{multiple-choice knapsack problem}):} Položky jsou rozděleny do skupin, přičemž do batohu může být z každé skupiny umístěna nejvýše jedna položka~\cite{knapsack-problems}.
\end{itemize}
Seznam výše obsahuje pouze několik vybraných variant problému batohu. 
Existuje totiž jeho mnoho dalších variant, například omezený problém batohu (\emph{bounded knapsack problem}), kvadratický problém batohu (\emph{quadratic knapsack problem}) a další~\cite{knapsack-problems,knapsack-problems-overview-I,knapsack-problems-overview-II}.

\section{Reprezentace problému batohu pro kvantově inspirované evoluční algoritmy}
V klasických evolučních algoritmech je řešení problému batohu reprezentováno jako bitový vektor, kde hodnota $1$ reprezentuje stav, kdy byla položka do batohu vložena a hodnota~$0$ značí nevložení položky do batohu.

V kvantově inspirovaných evolučních algoritmech je reprezentace řešení inspirována kvantovým bitem. 
Každé řešení problému batohu je utvářeno řetězcem $m$ kvantových bitů, přičemž každý qubit $i$ je složen z dvojice koeficientů $\alpha_i$ a $\beta_i$ dle rovnice~\ref{eq:quantum-representation}, kde $\left| \alpha_i \right|^2$ dle rovnice~\ref{eq:psi=a0+b1} představuje pravděpodobnost, že položka nebude do batohu vložena, zatímco $\left| \beta_i \right|^2$ udává pravděpodobnost jejího vložení. 
Tyto koeficienty musí splňovat normalizační podmínku~\ref{eq:a2+b2=1}.
Tento řetězec kvantových bitů reprezentuje superpozici všech možných řešení, přičemž jeho délka $m$ odpovídá počtu položek v instanci problému.

Tato kvantová reprezentace řešení určuje pravděpodobnosti vložení či nevložení položky do batohu. 
Aby bylo možné získat konkrétní binární řešení $x = \begin{pmatrix} x_1 & x_2 & \dots & x_m \end{pmatrix}$, je nutné provést pozorování kvantového stavu, při kterém je každý bit $x_i$ pro $i = 1,2,\dots\,m$ vektoru $x$ získán tak, že je vygenerováno náhodné číslo $r \in \langle 0, 1 \rangle$. 
Pokud je splněna podmínka $r < \left| \beta_i \right|^2$ je na $i$-té pozici binárního řetězce nastavena $1$, jinak je nastavena~$0$. 
Tento postup je popsán v~algoritmu~\ref{alg:make}.

\begin{algorithm}[ht!]
    \caption{Proces generování binárního řešení problému batohu}
    \label{alg:make}
    $i \gets 0$\;
    \While{$i < m$}{
        $i \gets i + 1$\;
        $r \gets  \text{random}[0,1]$\;
        \uIf{$r < \left| \beta_i \right|^2$}{
            $x_i \gets 1$\;
        }
        \Else{
            $x_i \gets 0$\;
        }
    }
\end{algorithm}

Po provedeném pozorování může řešení $x$ porušovat kapacitní omezení~\ref{eq:knapsack-capacity} batohu a~proto je nutné jej opravit. 
Proces opravy je popsán algoritmem~\ref{alg:repair}, přičemž algoritmus nejdříve určí zda bylo překročeno kapacitní omezení batohu. 
V případě jeho překročení jsou odstraňovány položky z batohu do té doby, dokud je batoh přeplněn. 

Po opravě případného překročení kapacitního omezení batohu v něm vzniká místo, které může být opět zaplněno. 
Následuje tedy cyklus, který vybírá položky a vkládá je do batohu, dokud batoh nepřekročí svoji kapacitu. 
V závěrečném kroku algoritmus odstraní poslední přidanou položku do batohu, aby zajistil nepřekročení jeho kapacity~\cite{qiga}.
\begin{algorithm}[ht!]
    \caption{Obecný proces opravy nevalidního řešení}
    \label{alg:repair}
    \textit{knapsack-overfilled} $\gets$ \textbf{nepravda}\;
    \If{$\sum_{i=1}^{m} w_i x_i > C$}{
        \textit{knapsack-overfilled} $\gets$ \textbf{pravda}\;
    }
    \While{\textit{knapsack-overfilled}}{
        Výběr $i$-té položku batohu\;
        $x_i \gets 0$\;
        \If{$\sum_{i=1}^{m} w_i x_i \leq C$}{
            \textit{knapsack-overfilled} $\gets$ \textbf{nepravda}\;
        }
    }
    \While{\textbf{not} \textit{knapsack-overfilled}}{
        Výběr $j$-té položku batohu\;
        $x_j \gets 1$\;
        \If{$\sum_{i=1}^{m} w_i x_i > C$}{
            \textit{knapsack-overfilled} $\gets$ \textbf{pravda}\;
        }
    }
    $x_j \gets 0$\;
\end{algorithm}

Případně je možné místo náhodného výběru položek využít princip, kdy jsou voleny položky na základě poměru ceny a váhy. 

Při experimentech byla využita místo obecného procesu opravy řešení heuristika, jenž místo náhodného odstraňování a přidávání položek volí vždy tu položku, jejíž poměr váha/hodnota je nejnevýhodnější pro odstranění nebo nejvýhodnější pro zpětné doplnění. 
Tento postup je popsán algoritmem~\ref{alg:repair-updated}.
\begin{algorithm}[ht]
    \caption{Proces opravy nevalidního řešení pomocí poměru váha/hodnota}
    \label{alg:repair-updated}
    \While{$\sum_{i=1}^m w_i x_i > C$}{
      Pro každé $i$ se spočte $\displaystyle r_i \gets \frac{w_i}{v_i}\,x_i$\;  
      $k \gets \arg\max_i r_i$\;  
      $x_k \gets 0$\;  
    }
    \While{$\sum_{i=1}^m w_i x_i \le C$}{
      Pro každé $i$ se spočte $\displaystyle r_i \gets \frac{w_i}{v_i}\,(1 - x_i)$\;  
      $k \gets \arg\max_i t_i$\;  
      $x_k \gets 1$\;  
    }
    $x_k \gets 0$\;
    \Return{$x$}\;
\end{algorithm}
Algoritmus zajišťuje, že výsledné řešení bude vždy splňovat kapacitní omezení a zároveň minimalizuje ztrátu kvality řešení. 
Nejprve jsou odstraňovány položky s nejhorším poměrem váha/hodnota a následně doplňuje položky s nejlepším poměrem, čímž je efektivně opraven kvantový chromozom. 

\section{Návrh experimentů}\label{sec:experiments-design}
Tato sekce podrobně popisuje návrh a parametrizaci experimentů, které slouží k analýze a vyhodnocení ideálních ladících parametrů jednotlivých algoritmů. 
Dále slouží k porovnání kvantově inspirovaných algoritmů mezi sebou a rovněž k porovnání s jejich klasickými variantami a to vše při řešení různých velikostí instancí problému batohu. 
Dále v sekcích proměnná $j$ označuje $j$-tého jedince v populaci, $i$ odpovídá $i$-té položce batohu a $t$ reprezentuje aktuální iteraci evolučního procesu. 

V tabulce \ref{tab:experiments-design} jsou uvedeny použité datové sady, velikosti instancí datových sad $n$, počet nezávislých opakování každého nastavení algoritmu a počet evaluací fitness funkce. 
Datové sady jsou rozděleny do dvou skupin:
\begin{itemize}
    \item datové sady čítající nízký počet položek problému batohu, tj. 100, 250 a 500\,\footnote{Zdroj: \url{https://github.com/mjBM/Quantum-Evolutionary-Algorithm-Knapsack-Python-}.} a
    \item datové sady s vysokým počtem položek problému batohu, respektive 1\,000, 2\,000, 5\,000 a 10\,000\,\footnote{Zdroj: \url{http://artemisa.unicauca.edu.co/~johnyortega/instances_01_KP/}.}.
\end{itemize}
Počet evaluací fitness byl pro většinu experimentů nastaven na $10\,000$, ale u největších instancí $n = 10\,000$ byly navíc provedeny běhy se $100\,000$ evaluacemi za účelem posouzení chování algoritmů v případě, že jim byl poskytnut prostor pro delší výpočet.  

\begin{table}[ht]
    \centering
    \begin{tabular*}{\linewidth}{@{\extracolsep{\fill}} 
        l
        c
        c
        c
        c
      @{}}
      \toprule
      \textbf{Datová sada} 
        & \textbf{Instance \(n\)} 
        & \textbf{Optimum} 
        & \textbf{Evaluace}
        & \textbf{Opakování} \\
      \midrule
      \midrule
      \multicolumn{5}{@{}l}{\textbf{Malé instanční sady}} \\
      \makecell[c]{---}
        &   100 & 3\,967 & 10\,000 & 30 \\
      \makecell[c]{---}
        &   250 & 10\,424 & 10\,000 & 30 \\
      \makecell[c]{---}
        &   500 & 20\,925 & 10\,000 & 30 \\
      \midrule
      \multicolumn{5}{@{}l}{\textbf{Velké instanční sady}} \\
      knapPI\_1\_1000\_1000\_1
        & 1\,000 & 54\,503 & 10\,000 & 30 \\
      knapPI\_1\_2000\_1000\_1
        & 2\,000 & 110\,625 & 10\,000 & 30 \\
      knapPI\_1\_5000\_1000\_1
        & 5\,000 & 276\,457 & 10\,000 & 30 \\
      knapPI\_1\_10000\_1000\_1
        &10\,000 & 563\,647 & 10\,000 / 100\,000 & 30 \\
      \bottomrule
    \end{tabular*}
    \caption{Přehled datových sad, optimálních hodnot a parametrů experimentů}
    \label{tab:experiments-design}
\end{table}
Pevně dané parametry evaluací a počet nezávislých opakování jednotlivých nastavení algoritmu umožňují vyhodnotit vliv velikosti problému, počtu jedinců v populaci a parametrů algoritmů na rychlost konvergence a kvalitu dosažených řešení. 

Následuje popis jednotlivých testovaných algoritmů a jejich nastavení, přičemž výsledky samotných experimentů budou prezentovány a doplněny o komentář v kapitole~\ref{chapt:experiments}
  
\subsection*{Kvantově inspirovaný genetický algoritmus}
Kvantově inspirovaný genetický algoritmus, jenž byl popsán v sekci~\ref{sec:qiga} společně s jeho pseudokódem~\ref{alg:ea-algo}, je možné ladit hodnotu rotace $\Delta\theta_{j_i}$ pravděpodobnostních koeficientů $\alpha_{j_i}\left(t\right)$ a~$\beta_{j_i}\left(t\right)$ po jednotkové kružnici, vizte obrázek~\ref{fig:qiga-rotation-gate}.
Experimentálně byla vyhodnocena závislost parametru rotace $\Delta\theta_{j_i}$ na velikosti populace tím způsobem, že každá hodnota rotace byla pro danou instanci $n$ spuštěna se všemi vybranými velikostmi populace, přičemž jednotlivá nastavení jsou zanesena v tabulce~\ref{tab:qiga-all-instance}.
\begin{table}[ht]
    \centering
    \begin{tabularx}{\linewidth}{@{} 
        l 
        p{0.45\linewidth} 
        p{0.45\linewidth} 
      @{}}
      \toprule
      \textbf{Instance $n$} & \textbf{Rotace $\Delta\theta_{j_i}$} & \textbf{Velikost populace} \\
      \midrule
      100  
        & 0,002, 0,01, 0,05, 0,1, 0,2, 0,5, 1, 2
        & 1, 5, 10, 20, 30, 40, 50, 100 \\[1ex]
      250  
        & 0,002, 0,01, 0,05, 0,1
        & 1, 5, 10, 20 \\[1ex]
      500  
        & 0,002, 0,01, 0,05, 0,1
        & 1, 5, 10, 20 \\[1ex]
      1000 
        & 0,002
        & 1 \\[1ex]
      2000 
        & 0,002
        & 1 \\[1ex]
      5000 
        & 0,002
        & 1 \\[1ex]
      10000 
        & 0,002
        & 1 \\
      \bottomrule
    \end{tabularx}
    \caption{Parametry testování algoritmu \emph{QIGA}}
    \label{tab:qiga-all-instance}
\end{table}
Vzhledem k tomu, že \emph{QIGA} nedisponuje mechanismem pro vzájemnou interakci jedinců v populaci, je vždy nejefektivnější jednočlenná populace, jelikož daný jedince bude vystaven největšímu počtu generací evolučního procesu. 
Přesto bylo experimentováno i s většími velikostmi populací, aby bylo možné zhodnotit do jaké míry velikost populace ovlivňuje rychlost konvergence a kvalitu výsledků.

Výsledky experimentů pro \emph{QIGA} jsou vyhodnoceny v následující kapitole v sekci~\ref{sec:exp-qiga}.

\subsection*{Kvantově inspirované simulované žíhání}
Algoritmus pro kvantově inspirované simulované žíhání představený v sekci~\ref{sec:qisa} společně s~odpovídajícím pseudokódem~\ref{alg:qisa} představený v článku~\cite{qisa} má za cíl minimalizaci energie.
Za účelem jeho aplikace na řešený problém batohu je potřeba provést úpravu některých částí algoritmu tak, aby docházelo k maximalizaci hodnot řešení.

Úprava algoritmu spočívala v obracení nerovnostni v akceptační podmínce~\ref{eq:qisa-if}, respektive:
\begin{equation*}
    e_j\left(t\right) > e_j\left(t-1\right).
\end{equation*}
Společně s úpravou podmínky byl modifikován výpočet faktoru $b$ tak:
\begin{equation*}
    b = \exp\left(\frac{e_j\left(t\right) - e_j\left(t-1\right)}{T_t}\right),
\end{equation*}
aby odpovídal směru optimalizace problému. 

Jedním z parametrů \emph{QISA} je počáteční teplota algoritmu, která je získána na základě vzorce~\ref{eq:qisa-T0}, jenž je však vhodný pro řešení numerických problémů. 
Pro kombinatorické problémy, kde řešení mohou být reprezentovány diskrétními objekty, je třeba provést úpravu výpočtu počáteční teploty. 
Nechť $f\left(x\right)$ je fitness funkce, pak výpočet teploty $T_0$ probíhá následovně:
\begin{equation}\label{eq:qisa-std}
    T_0 = \sigma = \sqrt{\frac{1}{N}\sum_{i=1}^{N}\left( f\left(x_i\right) - \bar{f}\right)^2},
\end{equation}
\begin{itemize}
    \item $x_i$ jsou jednotlivá náhodně vygenerovaná řešení,
    \item $\bar{f}$ je průměr hodnot fitness funkcí $f\left(x_i\right)$ a
    \item $N$ je počet vygenerovaných řešení (zde $N=1000$).
\end{itemize}
Tento postup zajišťuje využití \emph{QISA} i při řešení kombinatorického problému batohu~\cite{qisa,FundamentalsOfProbability}. 

Další parametry využívané v \emph{QISA} souvisí s dvěma popsanými metodami zahřívací funkce $h\left(T_t\right)$, respektive se jedná o konstantní zahřívání a sigmoidní zahřívání popsané po řadě rovnicemi~\ref{eq:qisa-const} a~\ref{eq:qisa-sigmo}. 
Hodnoty parametrů těchto metod byly zvoleny na základě článku~\cite{qisa} a~jsou přehledně zaneseny v tabulce~\ref{tab:qisa-heating-params}, kde proměnná $n$ reprezentuje velikost instance problému batohu.
\begin{table}[ht]
    \centering
    \begin{tabular}{lcc}
      \toprule
      \multirow{2}{*}{\textbf{Parametr}}  & \multicolumn{2}{c}{\textbf{Zahřívací funkce}} \\
      \cmidrule(lr){2-3}
            & \textbf{Konstantní}  & \textbf{Sigmoidní} \\
      \midrule
      $w$   & $\frac{9}{n}$        & ---                \\[1ex]
      $w_1$ & ---                  & $50$               \\[1ex]
      $w_2$ & ---                  & $0,2$              \\[1ex]
      $w_3$ & ---                  & $\frac{9}{n}$      \\[1ex]
      $w_4$ & ---                  & $\frac{9}{n}$      \\
      \bottomrule
    \end{tabular}
    \caption{Hodnoty parametrů zahřívacích funkcí}
    \label{tab:qisa-heating-params}
\end{table}

Parametry, jimiž bylo možné dále ladit algoritmus \emph{QISA}, zahrnovaly počet jedinců v populaci, výběr zahřívací funkce spolu se selekcí chladícího plánu a míry ochlazování teploty v průběhu iterací algoritmu. 
Experimentálně byly vyhodnoceny vzájemné závislosti těchto parametrů pomocí jejich kombinací při různých instancích problému batohu, přičemž hodnoty použitých parametrů jsou zaneseny v tabulce~\ref{tab:qisa-all-instances}.

\begin{table}[ht]
  \centering
  \begin{tabularx}{\textwidth}{@{} 
      c 
      >{\centering\arraybackslash}p{0.20\linewidth}
      >{\centering\arraybackslash}p{0.15\linewidth}
      >{\centering\arraybackslash}p{0.25\linewidth}
      >{\centering\arraybackslash}p{0.15\linewidth}
    @{}}
    \toprule
    \makecell[c]{\textbf{Instance}\\\textbf{$n$}} 
      & \makecell[c]{\textbf{Zahřívací}\\\textbf{funkce}} 
      & \makecell[c]{\textbf{Chladící}\\\textbf{schéma}}
      & \makecell[c]{\textbf{Míra}\\\textbf{ochlazování $\eta$}}
      & \makecell[c]{\textbf{Velikost}\\\textbf{populace}} \\
    \midrule
    \multirow{2}{*}{\makecell[c]{100}}  
      & \multirow{2}{=}{\makecell[l]{sigmoid, constant}}
      & exp, lin
      & 0,90, 0,95, 0,98, 0,99
      & \multirow{2}{=}{\makecell[l]{1, 5, 10, 20, 30,\\40, 50, 100}} \\ 
      \cmidrule(lr){3-4}
      & 
      & \makecell[c]{log, rec-log}
      & --- 
      & \\[1ex]
      \multirow{2}{*}{\makecell[c]{250}} 
      & \multirow{2}{=}{\makecell[l]{sigmoid, constant}}
      & exp, lin
      & 0,90, 0,95, 0,98, 0,99
      & \multirow{2}{=}{\makecell[l]{1, 5, 10, 20, 30,\\40, 50, 100}} \\ 
      \cmidrule(lr){3-4}
      & 
      & \makecell[c]{log, rec-log}
      & --- 
      & \\[1ex]
    \multirow{2}{*}{\makecell[c]{500}} 
      & \multirow{2}{=}{\makecell[l]{sigmoid, constant}}
      & exp, lin
      & 0,90, 0,95, 0,98, 0,99
      & \multirow{2}{=}{\makecell[l]{1, 5, 10, 20, 30,\\40, 50, 100}} \\ 
      \cmidrule(lr){3-4}
      & 
      & \makecell[c]{log, rec-log}
      & --- 
      & \\[1ex]
    1000  
      & sigmoid
      & \makecell[c]{rec-log}
      & --- 
      & 1 \\[1ex]
    2000  
      & sigmoid
      & \makecell[c]{rec-log}
      & --- 
      & 1 \\[1ex]
    5000  
      & sigmoid
      & \makecell[c]{rec-log}
      & --- 
      & 1 \\[1ex]
    10000  
      & sigmoid
      & \makecell[c]{rec-log}
      & --- 
      & 1 \\
    \bottomrule
  \end{tabularx}
  \caption{Parametry testování algoritmu \emph{QISA}}
  \label{tab:qisa-all-instances}
\end{table}

Obdobně jako \emph{QIGA}, i \emph{QISA} principiálně nevyužívá mechanismus vzájemné interakce jedinců, přesto však bylo experimentováno s různými velikostmi populací za účelem zhodnocení vlivu počtu jedinců na kvalitu řešení a rychlost konvergence. 

Hodnoty pro ochlazování byly vybrány na základě článku~\cite{qisa}, který uvažoval jediný chladící plán, respektive:
\begin{itemize}
    \item exponenciální (\emph{exponential\,--\,exp})~\cite{qisa,sa-cooling}
    \begin{equation}\label{eq:qisa-exp}
        T_t = T_0 \cdot \eta^t \quad\Longleftrightarrow\quad T_t = T_{t-1} \cdot \eta,
    \end{equation}
\end{itemize}
přičemž tento plán byl v rámci experimentů doplněn o další tři varianty:
\begin{itemize}
    \item lineární (\emph{linear\,--\,lin})~\cite{sa-cooling}
    \begin{equation}\label{eq:qisa-lin}
        T_t = T_0 - \eta \cdot t \quad\Longleftrightarrow\quad T_t = T_{t-1} - \eta
    \end{equation}
    \item logaritmický (\emph{logarithmic\,--\,log})~\cite{metaheuristics,sa-cooling, banchs_simulated_annealing}
    \begin{equation}\label{eq:qisa-log}
        T_t = \frac{T_0}{\ln\left(t+1\right)}
    \end{equation}
    \item rekurzivně-logaritmický (\emph{recursive-logarithmic\,--\,rec-log})
    \begin{equation}\label{eq:qisa-rec-log}
        T_t = \frac{T_{t-1}}{\ln\left(t+1\right)}
    \end{equation}
\end{itemize}
V exponenciálním a lineárním chladícím plánu proměnná $\eta$ označuje míru ochlazování, které však logaritmické a rekurzivně-logaritmické plánování nevyužívá.  
Poslední zmíněný chladící plán je vlastním rozšířením, protože na rozdíl od logaritmického schématu používající statickou počáteční teploty $T_0$, pracuje s dynamicky se měnící teplotou, jež je definována rekurzivně, přičemž nová hodnota $T_t$ se vypočte z předcházející teploty $T_{t-1}$.

Výsledky experimentů pro \emph{QISA} jsou vyhodnoceny v následující kapitole v sekci~\ref{sec:exp-qisa}.

\subsection*{Kvantová evoluce roje}
Kvantová evoluce roje, popsaná v sekci~\ref{sec:qse} a pseudokódem v algoritmu~\ref{alg:qse} neobsahuje možnost nastavení ladících parametrů neboť koeficienty pro výpočet rychlosti z rovnice~\ref{eq:qse-velocity} jsou dány tabulkou~\ref{tab:qse-params}, kde 
koeficienty $\omega$, $c_1$ a $c_2$ splňují konvergenční podmínku částic:
\begin{equation*}
    \omega > \frac{c_1 + c_2}{2} - 1.
\end{equation*}
Rovněž budou částice rychleji konvergovat ke globálnímu optimu roje než k lokálnímu optimu každé částice neboť platí nerovnost $c_2 > c_1$.

\begin{table}[ht]
    \centering
    \begin{tabular}{c c l}
      \toprule
      \textbf{Koeficient} & \textbf{Hodnota} & \makecell[c]{\textbf{Význam}} \\
      \midrule
      $\chi$   & $0,99$    & koeficient zúžení (omezení rychlosti částice)       \\
      $\omega$ & $0,7298$  & inerciální váha (vyvážení explorace a exploitace)   \\
      $c_1$    & $1,42$    & kognitivní koeficient (vliv nejlepší osobní pozice) \\
      $c_2$    & $1,57$    & sociální koeficient (vliv nejlepší globální pozice) \\
      \bottomrule
    \end{tabular}
    \caption{Význam a hodnoty parametrů \emph{QSE}~\cite{PSO-X,PSO-c1c2w}}
    \label{tab:qse-params}
\end{table}

Protože ladící parametry algoritmu byly dány, byl v rámci experimentů srovnán vliv počtu částic v populaci společně s různými počátečními rychlostmi částic, neboť originální článek~\cite{qse} explicitně neuváděl výchozí hodnotu pro inicializaci počáteční rychlosti jednotlivých částic, vizte tabulku~\ref{tab:qse-all-instances}. 

\begin{table}[ht]
    \centering
    \begin{tabular}{ 
        c 
        >{\centering\arraybackslash}c
        >{\centering\arraybackslash}c
      }
      \toprule
      \textbf{Instance $n$} 
        & \textbf{Počáteční rychlost}
        & \textbf{Velikost populace}  \\
      \midrule
      100
        & 0, 1, 2, 5, 10, 25, 50, 100
        & 1, 5, 10, 20, 30, 40, 50, 100  \\[1ex]
      250  
        & 0, 1, 2, 5, 10, 25, 50, 100
        & 1, 5, 10, 20, 30, 40, 50, 100  \\[1ex]
      500  
        & 0, 1, 2, 5, 10, 25, 50, 100
        & 1, 5, 10, 20, 30, 40, 50, 100  \\[1ex]
      1000 
        & 1 
        & 5  \\[1ex]
      2000 
        & 1 
        & 5  \\[1ex]
      5000 
        & 1 
        & 5  \\[1ex]
      10000
        & 1 
        & 5  \\
      \bottomrule
    \end{tabular}
    \caption{Parametry testování algoritmu \emph{QSE}}
    \label{tab:qse-all-instances}
\end{table}

Na rozdíl od algoritmů \emph{QIGA} a \emph{QISA}, kde nedochází ke komunikaci jedinců v populaci, algoritmus \emph{QSE} tento mechanismus pro vzájemnou interakci jedinců využívá, což umožňuje sdílení informací o kvalitě nalezených řešení mezi jedinci. 

Výsledky experimentů pro \emph{QSE} jsou vyhodnoceny v následující kapitole v sekci~\ref{sec:exp-qse}.

\subsection*{Kvantově inspirovaná optimalizace rojem částic}
Vlastní algoritmus kvantově inspirované optimalizace rojem částic, vycházející z kombinace principů z \emph{QIGA} a z \emph{QSE} a popsaný v sekci~\ref{sec:qipso} společně s jeho pseudokódem v algoritmu~\ref{alg:qipso}, již nevyužívá pevně daných hodnot koeficientů jako u \emph{QSE}. 

Proto byl experimentálně ověřen vliv různých kombinací počáteční rychlosti jednotlivých částic, kognitivního koeficientu $c_1$, sociálního koeficientu $c_2$, koeficientu zúžení $\omega$ při různých velikostech populace. 
Hodnoty jednotlivých parametrů, jenž byly podrobeny testování, jsou zaneseny v tabulce~\ref{tab:qipso-all-params}.

\begin{table}[ht]
    \centering
    \begin{tabularx}{\linewidth}{@{} 
        >{\centering\arraybackslash}p{0.11\linewidth}
        >{\centering\arraybackslash}p{0.23\linewidth}
        >{\centering\arraybackslash}p{0.1\linewidth}
        >{\centering\arraybackslash}p{0.1\linewidth}
        >{\centering\arraybackslash}p{0.141\linewidth}
        >{\centering\arraybackslash}p{0.18\linewidth}
      @{}}
      \toprule
      \makecell[c]{\textbf{Instance}\\\textbf{$n$}} 
        & \makecell[c]{\textbf{Počáteční}\\\textbf{rychlost}}
        & \(\boldsymbol{c_1}\)
        & \(\boldsymbol{c_2}\)
        & \(\boldsymbol{\omega}\)
        & \makecell[c]{\textbf{Velikost}\\\textbf{populace}} \\
      \midrule
      \makecell[c]{100}
        & \makecell[c]{0, 1, 2, 5, 10, 25, 50,\\100, 150, 200}
        & \makecell[c]{0,1, 0,25,\\0,5, 0,75}
        & \makecell[c]{0,1, 0,25,\\0,5, 0,75}
        & \makecell[c]{0,002, 0,01,\\0,05, 0,1, 0,2}
        & \makecell[c]{1, 5, 10, 20,\\30, 40, 50, 100}\\[1ex]
      \makecell[c]{250}
        & \makecell[c]{0, 1, 2, 5, 10, 25, 50,\\100, 150, 200}
        & \makecell[c]{0,5, 0,75}
        & \makecell[c]{0,1, 0,25 }
        & \makecell[c]{0,01, 0,05}
        & \makecell[c]{1, 5, 10, 20,\\30, 40, 50, 100}\\[1ex]
      \makecell[c]{500}
        & \makecell[c]{0, 1, 2, 5, 10, 25, 50,\\100, 150, 200}
        & \makecell[c]{0,1, 0,25,\\0,5, 0,75}
        & \makecell[c]{0,1, 0,25,\\0,5, 0,75}
        & \makecell[c]{0,002, 0,01,\\0,05, 0,1, 0,2}
        & \makecell[c]{1, 5, 10, 20,\\30, 40, 50, 100}\\[1ex]
      1000 
        & 50
        & 0,5 
        & 0,25
        & 0,01
        & 5 \\[1ex]
      2000 
        & 50
        & 0,5 
        & 0,25
        & 0,01
        & 5 \\[1ex]
      5000
        & 50
        & 0,5 
        & 0,25
        & 0,01
        & 5 \\[1ex]
      10000
        & 50
        & 0,5 
        & 0,25
        & 0,01
        & 5 \\
      \bottomrule
    \end{tabularx}
    \caption{Parametry testování algoritmu \emph{QIPSO}}
    \label{tab:qipso-all-params}
\end{table}

Obdobně jako u \emph{QSE}, i algoritmus \emph{QIPSO} využívá mechanismu vzájemné interakce částic v populaci, čímž je umožněno sdílení informací o kvalitě dosažených řešení. 

Výsledky experimentů pro \emph{QIPSO} jsou vyhodnoceny v následující kapitole v sekci~\ref{sec:exp-qipso}.

\subsection*{Klasická varianta GA, SA a PSO}
Pro srovnání s kvantově inspirovanými evolučními algoritmy bylo experimentováno i s jejich klasickými variantami a to konkrétně s:
\begin{itemize}
  \item genetickým algoritmem \emph{Genetic Algorithm\,--\,(GA)},
  \item simulovaným žíháním \emph{Simulated Annealing\,--\,(SA)} a
  \item částicovým systémem \emph{Particle Swarm Optimization\,--\,(PSO)}.
\end{itemize}
Ve všech třech přístupech jsou jednotliví jedinci v populaci reprezentováni pomocí binárního vektoru délky $n$, kde hodnota 1 na $i$-té pozici v chromozomu znamená vložení $i$-té položky do batohu a 0 její vynechání. 
Cílem všech algoritmů je opět maximalizovat celkovou hodnotu položek v batohu.

Výsledky experimentů jsou souhrnně vyhodnoceny v následující kapitole v sekci~\ref{sec:exp-ea}.

\subsubsection*{Genetický algoritmus}
Princip genetického algoritmu vychází z popisu obecného evolučního algoritmu ze sekce~\ref{sec:ea}. 
Experimenty uvažovaly následující varianty selekčních a variančních operátorů~\cite{ga-app}:
\begin{itemize}
    \item \textbf{Selekce rodičů:} Rodiče jsou vybíráni na základě tzv. vážené rulety. 
        Tato metoda spočívá v tom, že je každému jedinci přiřazena hodnota pravděpodobnosti, která je spočtena pomocí poměru fitness daného jedince a součtu fitness všech jedinců. 
        Při volbě rodičů jsou pak jedinci vybíráni na základě jejich pravděpodobností, respektive kvalitnější jedinci mají vyšší šanci dostat se do nové populace.
    \item \textbf{Křížení:} Rodiče jsou rozděleny do párů a na každou dvojici je aplikování uniformní křížení. 
        Při něm vznikají dva potomci, jejichž chromozomy jsou složeny z genů obou rodičů, přičemž každý gen je zděděn od jednoho z rodičů podle dané pravděpodobnosti.
    \item \textbf{Mutace:} Každý gen všech potomků je invertován s danou pravděpodobností.
    \item \textbf{Selekce přeživších:} Do nové populace je vybráno $n$ nejlepších jedinců ze sjednocené množiny rodičů a potomků. 
        Zároveň jsou zachováni i elitní jedinci z předcházející generace. 
\end{itemize}

Experimentálně byl ověřen vliv různých kombinací pravděpodobnosti mutace, pravděpodobnosti křížení, počtu elit a velikosti populace. 
Pro shrnutí nastavení jednotlivých parametrů vizte tabulku~\ref{tab:ga-all-params}.
\begin{table}[ht]
  \centering
  \begin{tabular}{c c c c c}
    \toprule
    \makecell[c]{\textbf{Instance}\\\textbf{$n$}} 
      & \makecell[c]{\textbf{Pravděpodobnost}\\\textbf{mutace}}
      & \makecell[c]{\textbf{Pravděpodobnost}\\\textbf{křížení}}
      & \textbf{Počet elit}
      & \makecell[c]{\textbf{Velikost}\\\textbf{populace}} \\
    \midrule
    \makecell[c]{100}
      & \makecell[c]{0,2, 0,4, 0,6,\\0,8, 1,0}
      & \makecell[c]{0,2, 0,4, 0,6,\\0,8, 1,0}
      & \makecell[c]{0, 1, 2, 3,\\4, 5}
      & \makecell[c]{1, 5, 10, 20, 30,\\40, 50, 100}\\[1ex]
    \makecell[c]{250}
      & \makecell[c]{0,005, 0,01, 0,15}
      & \makecell[c]{0,6, 0,7, 0,8, 0,9, 1}
      & \makecell[c]{1}
      & \makecell[c]{2}\\[1ex]
    \makecell[c]{500}
      & \makecell[c]{0,005, 0,01, 0,15}
      & \makecell[c]{0,6, 0,7, 0,8, 0,9, 1}
      & \makecell[c]{1}
      & \makecell[c]{2}\\
    \bottomrule
  \end{tabular}
  \caption{Parametry testování algoritmu \emph{GA}}
  \label{tab:ga-all-params}
\end{table}
Na rozdíl od kvantově inspirované varianty klasický genetický algoritmus využívá přímé interakce jedinců prostřednictvím selekce, křížení a mutace, čímž umožňuje sdílení informací o řešení napříč jedinci.


\subsubsection*{Simulované žíhání}
Simulované žíhání je metaheuristická metoda inspirovaná procesem chladnutí materiálu ve fyzice. 
Algoritmus pracuje s jediným řešením, které je v každé iteraci algoritmu náhodně modifikováno. 
Nové řešení je přijato v případě, že došlo ke zlepšení fitness, avšak pokud je nové řešení horší může být rovněž přijato, ale pouze s určitou pravděpodobností. 
Tato pravděpodobnost je u \emph{SA} dána teplotou, která se v průběhu času snižuje, což má za následek v počátečních fázích pravděpodobnějšího přijetí horšího řešení než v pozdějších fázích, což umožňuje předcházet uvíznutím v lokálních minimech~\cite{sa-app}. 

Obdobně jako u experimentů s \emph{QISA} i zde byl uvažován výpočet počáteční teploty na základě standardní odchylky dle vzorce~\ref{eq:qisa-std}, přičemž tato teplota je snižována v průběhu iterací pomocí různých chladících plánů, jež byly představeny vzorci~\ref{eq:qisa-exp},~\ref{eq:qisa-lin},~\ref{eq:qisa-log} a~\ref{eq:qisa-rec-log}.
Jednotlivé použité chladicí plány a jejich hodnoty parametru $\eta$ spolu s testovanými velikostmi populace, jimiž bylo \emph{SA} v průběhu experimentů laděno, jsou uvedeny v tabulce~\ref{tab:sa-all-instances}.
\begin{table}[ht]
  \centering
  \begin{tabular}{
    c 
    >{\centering\arraybackslash}c 
    >{\centering\arraybackslash}c 
    c
  }
  \toprule
    \makecell[c]{\textbf{Instance}\\\textbf{$n$}} 
      & \makecell[c]{\textbf{Chladící}\\\textbf{schéma}}
      & \makecell[c]{\textbf{Míra}\\\textbf{ochlazování $\eta$}}
      & \makecell[c]{\textbf{Velikost}\\\textbf{populace}} \\
    \midrule
    \multirow{2}{*}{100}  
      & exp, lin
      & 0,90; 0,95; 0,98; 0,99
      & \multirow{2}{*}{\makecell[c]{1, 5, 10, 20, 30, \\ 40, 50, 100}} \\ 
      \cmidrule(lr){2-3}
      & log, rec-log
      & ---
      & \\[1ex]
    \multirow{2}{*}{250} 
      & exp, lin
      & 0,90; 0,95; 0,98; 0,99
      & \multirow{2}{*}{\makecell[c]{1, 5, 10, 20, 30, \\ 40, 50, 100}} \\ 
      \cmidrule(lr){2-3}
      & log, rec-log
      & ---
      & \\[1ex]
    \multirow{2}{*}{500} 
      & exp, lin
      & 0,90; 0,95; 0,98; 0,99
      & \multirow{2}{*}{\makecell[c]{1, 5, 10, 20, 30, \\ 40, 50, 100}} \\ 
      \cmidrule(lr){2-3}
      & log, rec-log
      & ---
      & \\
    \bottomrule
  \end{tabular}
  \caption{Parametry testování algoritmu \emph{SA}}
  \label{tab:sa-all-instances}
\end{table}

Rovněž jako \emph{QIGA} a \emph{QISA}, i algoritmus \emph{SA} nevyužívá mechanismu vzájemné interakce jedinců, neboť pracuje pouze s jediným řešením v čase. 
Přesto však byly v experimentech uvažovány různé velikosti populací, a to za účelem zhodnocení rychlosti konvergence. 

\subsubsection*{Částicový systém}
Částicový systém se inspiruje sociálním chováním hejn ptáků a ryb, kdy každá částice v roji reprezentuje jedno řešení problému. 
Každá částice má vlastní rychlost, která koriguje pohyb částice v prostoru řešení a to na základě vlastní zkušenosti částice společně se sdílenou znalostí o nejlepší částici, přičemž každá ze zkušeností má při aktualizaci rychlosti danou váhu. 
Jedná se o tzv. kognitivní koeficient $c_1$ a sociální koeficient $c_2$~\cite{pso-app, PSO-c1c2w}. Dále experimenty uvažují i koeficient zúžení $\omega$, neboť redukovat rychlosti částic~\cite{PSO-X}. 

Přehled všech kombinací zmíněných koeficientů a velikostí populací pro jednotlivé instance je uveden v tabulce~\ref{tab:pso-all-params}.
\begin{table}[ht]
  \centering
  \begin{tabular}{c c c c c}
    \toprule
    \makecell[c]{\textbf{Instance}\\\textbf{$n$}} 
      & \(\boldsymbol{c_1}\)
      & \(\boldsymbol{c_2}\)
      & \(\boldsymbol{\omega}\)
      & \makecell[c]{\textbf{Velikost}\\\textbf{populace}} \\
    \midrule
    \makecell[c]{100}
      & \makecell[c]{0,5, 0,75, 1,0,\\1,25, 1,5}
      & \makecell[c]{0,5, 0,75, 1,0,\\1,25, 1,5}
      & \makecell[c]{0,5, 0,75, 1,0,\\1,25, 1,5}
      & \makecell[c]{1, 5, 10, 20, 30,\\40, 50, 100}\\[1ex]
    \makecell[c]{250}
      & \makecell[c]{1,25, 1,5}
      & \makecell[c]{0,5, 0,75}
      & \makecell[c]{1}
      & \makecell[c]{1, 5, 10, 20,\\30, 40, 50}\\[1ex]
    \makecell[c]{500}
      & \makecell[c]{1,25, 1,5}
      & \makecell[c]{0,5, 0,75}
      & \makecell[c]{1}
      & \makecell[c]{1, 5, 10, 20,\\30, 40, 50}\\
    \bottomrule
  \end{tabular}
  \caption{Parametry testování algoritmu \emph{PSO}}
  \label{tab:pso-all-params}
\end{table}

Algoritmus \emph{PSO} na rozdíl od metod \emph{SA}, \emph{QISA} a \emph{QIGA} využívá mechanismus kolektivního chování, díky čemuž dochází ke sdílení informací napříč částicemi, což umožňuje udržovat poměr mezi hledáním nových oblastí a zpřesňováním stávajících oblastí se slibnými řešeními. 

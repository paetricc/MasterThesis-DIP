\chapter{Struktura odevzdaných souborů}

\dirtree{%
.1 .
.2 DIP/\DTcomment{Složka se soubory pro překlad diplomové práce práce}.
.2 SRC/\DTcomment{Složka se zdrojovými soubory}.
.3 datasets/\DTcomment{Složka s instancemi problému batohu}.
.3 outputs/\DTcomment{Složky se získanými daty}.
.3 graphs/\DTcomment{Složka s vygenerovanými grafy}.
.3 scripts/\DTcomment{Složky se skripty}.
.3 src/\DTcomment{Složka se zdrojovými soubory}.
.4 plots/\DTcomment{Složka se zdrojovými soubory pro generování grafů}.
.4 qiea/\DTcomment{Složka se zdrojovými soubory algoritmů}.
.4 plots.py\DTcomment{Vstupní bod pro generování grafů}.
.4 qiea.py\DTcomment{Vstupní bod pro spuštění algoritmů}.
.3 Makefile\DTcomment{Makefile pro jednoduší práci s projektem}.
.3 README.md\DTcomment{Soubor popisující základní práci se zdrojovými programy}.
.2 thesis.pdf\DTcomment{Text diplomové práce}.
}

\chapter{Manuál}
Uvažujme kořenový adresář projektu. Základní nápovědu k programu lze zobrazit příkazem:
\begin{lstlisting}[style=bash]
    $ python3 src/qiea.py --help
\end{lstlisting}
Algoritmus lze spustit s výchozími hodnotami parametrů následovně:
\begin{lstlisting}[style=bash]
    $ python3 src/qiea.py qiga datasets/100.txt
\end{lstlisting}
Pro uložení výsledků do zvoleného výstupního adresáře:
\begin{lstlisting}[style=bash]
    $ python3 src/qiea.py qiga datasets/100.txt --output outputs/
\end{lstlisting}
V případě paralelního spouštění je možné připojovat výsledky k již existujícím výstupům:
\begin{lstlisting}[style=bash]
    $ python3 src/qiea.py qiga datasets/100.txt --output outputs/ --append_results
\end{lstlisting}
Pomocí přiloženého souboru \texttt{Makefile} lze snadno spouštět skripty pro experimenty.
\begin{lstlisting}[style=bash]
    $ make
\end{lstlisting}
Pro spuštění experimentů s konkrétním algoritmem:
\begin{lstlisting}[style=bash]
    $ make exp-qiga
\end{lstlisting}
A pro konkrétní instanci:
\begin{lstlisting}[style=bash]
    $ make exp-qiga-100
\end{lstlisting}
Ze získaných dat lze vygenerovat grafy příkazem:
\begin{lstlisting}[style=bash]
    $ make graphs
\end{lstlisting}
Pro generování grafů pouze pro vybrané algoritmy a instance:
\begin{lstlisting}[style=bash]
    $ python3 src/plots.py outputs/ --save_dir graphs/ --algorithm qiga qisa --instance 100 250
\end{lstlisting}

\section*{Seznam argumentů programu Quantum Inspired Evolutionary Algorithms (\texttt{qiea.py})}
\begin{itemize}
    \item \texttt{algorithm}\,--\,Typ algoritmu, který se má spustit. 
    \begin{itemize}
        \item Možnosti: \texttt{qiga}, \texttt{qisa}, \texttt{qse}, \texttt{qipso}.
    \end{itemize}
    \item \texttt{input}\,--\,Cesta k vstupnímu datasetu.
    \item \texttt{-{}-append\_results}\,--\,Pokud je nastaveno, výsledky budou připojovány do výstupního souboru po každém experimentu.
    \item \texttt{-{}-output}\,--\,Cesta k adresáři, kam budou uloženy výsledky.
    \item \texttt{-{}-experiments}\,--\,Počet nezávislých spuštění algoritmu (výchozí: \texttt{30}).
    \item \texttt{-{}-population}\,--\,Velikost populace (výchozí: \texttt{1}).
    \item \texttt{-{}-evaluations}\,--\,Počet vyhodnocení fitness funkce (výchozí: \texttt{10000}).
    \item \texttt{-{}-theta}\,--\,Úhel kvantové rotace pro QIEA (výchozí: \texttt{0,01}).
    \item \texttt{-{}-c1}\,--\,Kognitivní koeficient pro QIPSO (výchozí: \texttt{0,05}).
    \item \texttt{-{}-c2}\,--\,Sociální koeficient pro QIPSO (výchozí: \texttt{0,15}).
    \item \texttt{-{}-omega}\,--\,Parametr $\omega$ pro QIPSO (výchozí: \texttt{0,01}).
    \item \texttt{-{}-velocity}\,--\,Počáteční rychlost pro QIPSO a QSE. (výchozí: \texttt{100}).
    \item \texttt{-{}-cooling\_rate}\,--\,Rychlost ochlazování pro \emph{QISA} (výchozí: \texttt{0,98}).
    \item \texttt{-{}-cooling}\,--\,Metoda ochlazování pro \emph{QISA} (výchozí: \texttt{rec-logarithmic}). 
    \begin{itemize}
        \item Možnosti: \texttt{exponential}, \texttt{linear}, \texttt{logarithmic}, \texttt{rec-logarithmic}.
    \end{itemize}
    \item \texttt{-{}-observation}\,--\,Metoda tepelně-řízeného pozorování pro \emph{QISA}. (výchozí: \texttt{sigmoid}).
    \item \begin{itemize}
        \item Možnosti: \texttt{constant}, \texttt{sigmoid}.
    \end{itemize}
\end{itemize}

\section*{Seznam argumentů programu pro vykreslení grafů (\texttt{plots.py})}
\begin{itemize}
    \item \texttt{source\_dir}\,--\,Cesta ke zdrojové složce obsahující výsledky experimentů.
    \item \texttt{-{}-save\_dir}\,--\,Cesta, kam se mají uložit vykreslené grafy. Pokud není zadáno, grafy se pouze zobrazí.
    \item \texttt{-{}-algorithm}\,--\,Typ algoritmu, pro který se mají vykreslit výsledky. Lze zadat více hodnot oddělených mezerou.
    \begin{itemize}
        \item Možnosti: \texttt{qiga}, \texttt{qisa}, \texttt{qse}, \texttt{qipso}.
    \end{itemize}
    \item \texttt{-{}-instance}\,--\,Velikosti instancí problému, které se mají zpracovat. Lze zadat více hodnot oddělených mezerou.
\end{itemize}
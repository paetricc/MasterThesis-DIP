\chapter{Úvod}
Na každém kroku se v běžném životě setkáváme s problémy, jako je například plánování tras, správa financí nebo organizace času, a algoritmy nám pomáhají tato rozhodnutí automatizovat za pomoci výpočetní techniky. 
Ať už se jedná o nalezení nejrychlejší cesty v~navigační aplikaci nebo optimalizaci vysokoškolského rozvrhu, algoritmy se snaží najít optimální řešení z velkého počtu možností. 
Některé z nich spadají do skupiny problémů, ve které je nalezení nejlepšího možného řešení výpočetně velmi náročné. 
Takovéto řešení není možné nalézt v rozumném čase, a proto se využívají přístupy, které se snaží vyhledat dostatečně kvalitní řešení v přijatelném čase. 

Jedním z možných přístupů je počítání podle přírody, což je disciplína zaměřená na návrh algoritmů, které se inspirují přírodou a jejími procesy. 

Do podoblasti počítání podle přírody se řadí evoluční počítání, do kterého patří algoritmy, jež se inspirovaly biologickou evolucí, souhrnně označované jako evoluční algoritmy. 
Při svém výpočtu využívají evoluční mechanismy selekce, mutace a křížení za účelem efektivního prohledávání stavového prostoru. 
Přestože evoluční algoritmy našly uplatnění v~široké škále aplikací, například v návrhu neuronových sítí nebo v medicínských diagnostických systémech~\cite{ea-applications}, mají svá omezení v podobě uváznutí v lokálním minimu nebo pomalé konvergenci k řešení. 
Je tedy nutné udržovat správnou rovnováhu mezi průzkumem stavového prostoru a zužitkováním oblastí se slibnými výsledky.

Do výpočetních metod inspirovaných přírodou spadá i fyzikální počítání, které využívá principy z různých oblastí fyziky. 
Jedním z možných příkladů je kvantová mechanika, která pracuje s fenoménem superpozice, kdy částice může být v současné době ve více stavech, dokud není pozorována. 
Tento princip ilustroval Erwin Schrödinger na slavném příkladu kočky v~krabici, která je zároveň živá i mrtvá, dokud není krabice otevřena. 
Na podobném principu umožňují kvantové výpočty zpracovávat více možností současně, což zrychluje řešení úloh.
Jelikož kvantové počítače nejsou běžně dostupné, lze využít některé z kvantových principů i na klasických počítačích, což dalo podnět k realizaci kvantově inspirovaných algoritmů. 

Kombinace principů z obou zmíněných podoblastí počítaní podle přírody vedla k~vzniku kvantově inspirovaných evolučních algoritmů. 
Algoritmy využívají principy kvantové mechaniky k tomu, aby byl evoluční výpočet při prozkoumávání stavového prostoru efektivnější a současně nacházel kvalitnější řešení. 

Cílem práce je provést analýzu různých typů kvantově inspirovaných evolučních algoritmů při řešení problému batohu a následně zhodnotit získané výsledky a porovnat je s~výsledky klasických evolučních metod a uvést případná možná omezení a nedostatky navržených přístupů.

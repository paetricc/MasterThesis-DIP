\chapter{Úvod}
Na každém kroku se v běžném životě setkáváme s problémy, jako je například plánování tras, správa financí nebo organizace času, a algoritmy nám pomáhají tato rozhodnutí automatizovat pomocí výpočetní techniky. 
Ať už se jedná o nalezení nejrychlejší cesty v~navigační aplikaci nebo optimalizaci vysokoškolského rozvrhu, algoritmy se snaží najít optimální řešení z velkého množství možností. 
Některé z nich však spadají do skupiny problémů, u nichž je nalezení nejlepšího možného řešení výpočetně velmi náročné. 
Takové řešení není možné nalézt v rozumném čase, a proto se využívají přístupy, které se snaží vyhledat dostatečně kvalitní řešení v přijatelném čase. 

Jedním z těchto přístupů je počítání podle přírody, což je disciplína zaměřená na návrh algoritmů, které se inspirují přírodou a jejími procesy. 

Do podoblasti počítání podle přírody spadá evoluční počítání, které zahrnuje algoritmy inspirované biologickou evolucí, souhrnně označované jako evoluční algoritmy. 
Tyto algoritmy při výpočtu využívají evoluční mechanismy selekce, mutace a křížení za účelem efektivního prohledávání stavového prostoru. 
Přestože evoluční algoritmy nalezly uplatnění v~široké škále aplikací, například při návrhu neuronových sítí nebo v medicínských diagnostických systémech~\cite{ea-applications}, potýkají se s určitými omezeními, jako je uváznutí v lokálním minimu/maximu nebo pomalé konvergenci k řešení. 
Je proto důležité udržovat správnou rovnováhu mezi průzkumem stavového prostoru a zužitkováním oblastí se slibnými výsledky.

Do výpočetních metod inspirovaných přírodou spadá také fyzikální počítání, které využívá principy z různých oblastí fyziky. 
Jedním z příkladů je kvantová mechanika, která pracuje s fenoménem superpozice, kdy částice může být v současné době ve více stavech, dokud není pozorována. 
Tento princip ilustroval Erwin Schrödinger ve slavném myšlenkovém experimentu s kočkou v~krabici, která je zároveň živá i mrtvá, dokud není krabice otevřena. 
Na podobném principu umožňují kvantové výpočty zpracovávat více možností současně, což může výrazně urychlit řešení některých výpočetně náročných úloh. 
Vzhledem k tomu, že kvantové počítače zatím nejsou běžně dostupné, lze využít některé z kvantových principů i na klasických počítačích, což vedlo k vývoji kvantově inspirovaných algoritmů. 

Kombinace principů z obou zmíněných podoblastí počítaní podle přírody vedla ke~vzniku kvantově inspirovaných evolučních algoritmů. 
Tyto algoritmy využívají principy kvantové mechaniky s cílem zvýšit efektivitu evolučního počítání při prohledávaní prostoru řešení. 

Cílem práce je analyzovat různé typy kvantově inspirovaných evolučních algoritmů při řešení problému batohu, vyhodnotit získané výsledky a porovnat je s~výsledky klasických evolučních metod. 
Součástí analýzy je také identifikace případných omezení a nedostatků zkoumaných přístupů. 
Na základě provedeného zhodnocení je dále navržen, implementován a posouzen vlastní kvantově inspirovaný evoluční algoritmus. 


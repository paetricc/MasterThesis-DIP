\chapter{Východiska pro kvantově inspirované evoluční algoritmy}
Přírodní procesy představují významný zdroj inspirace při návrhu algoritmů, neboť se ukázalo, že jejich využitím lze vytvořit účinné metody pro efektivní řešení složitých výpočetních úloh.
Podněty pro tvorbu algoritmů lze čerpat nejen z biologických systémů a procesů, ale také z různých oblastí fyziky. 
Oba tyto přístupy je navíc možné kombinovat a vytvářet algoritmy, které spojují principy biologické evoluce s fyzikálními principy~\cite{NaturalComputing,NaturalComputing-handbook}.

Tato kapitola nejprve v krátkosti představí základní oblasti přírody, jimiž je možné se inspirovat ve výpočetních metodách. 
Dále budou ve stručnosti uvedeny základy evolučních algoritmů a v závěru kapitoly budou popsány vybrané principy kvantové fyziky a kvantové mechaniky, které slouží jako teoretický základ pro návrh kvantově inspirovaných evolučních algoritmů. 

\section{Algoritmy inspirované přírodou}
Počítání podle přírody (\emph{natural computing}) se zaměřuje na objevování fundamentálních zákonů a mechanismů, na jejichž principu fungují procesy v přírodě, a na jejich následné využití při řešení různých výpočetních problémů. 
To vede k návrhu nových způsobů výpočtu a vývoji alternativních řešení různých typů úloh. 
Aby bylo možné efektivně modelovat systémy s velkým počtem prvků, je nutné zvolit vhodnou úroveň abstrakce, jež je závislá na charakteru řešené úlohy, neboť algoritmy inspirované přírodou vycházejí ze zjednodušeného pojetí přírodních procesů. 
Abstrakce zároveň umožňuje zdůraznit klíčové aspekty systému, jež jsou zásadní pro jeho reprodukovatelnost a analýzu vývojových procesů. 
Zvolená úroveň abstrakce je závislá na charakteru řešeného problému, typu modelovaného jevu a cílů simulace. 
Počítání podle přírody spojuje znalosti z teoretické a experimentální teorie fungování přírody společně s empirickými pozorováními za účelem návrhu nových způsobů pro řešení různých problémů, vizte obrázek~\ref{fig:natural-computing-fields}~\cite{FundamentalNatural}.

Počítání podle přírody je možné rozdělit do následujících tří oblastí~\cite{FundamentalNatural}:
\begin{enumerate}
    \item \textbf{Výpočty inspirované přírodou}
        vycházejí z předpokladu, že přírodní procesy dokáží efektivně řešit složité problémy, přičemž tyto přístupy lze rozdělit na dva hlavní směry. 
        První vychází z teoretických modelů, pomocí nichž lze modelovat přírodní jevy, což pomáhá k lepšímu porozumění fungování různých procesů v přírodě. 
        Druhým směrem je návrh algoritmů, které pomáhají řešit složité úlohy, u nichž tradiční přístupy selhávají, přičemž tyto algoritmy využívají určitou úroveň abstrakce nad vybranými přírodními procesy.
        Mezi nejvýznamnější oblasti výpočtů inspirovaných přírodou patří~\cite{FundamentalNatural,NaturalComputing}:
        \begin{itemize}
            \item umělé neuronové sítě (\emph{artificial neural networks}) inspirované nervovými systémy,
            \item evoluční algoritmy (\emph{evolutionary algorithms}) vycházející z Darwinovy teorie evoluce,
            \item inteligence roje (\emph{swarm intelligence}) založené na kolektivním chování společenských organismů a
            \item umělé imunitní systémy (\emph{artificial immune systems}) vycházející z principů fungování imunitního systému.
        \end{itemize}
        \begin{figure}[ht!]
            \centering
            \includegraphics[width=0.75\textwidth]{NaturalComputing-fields.pdf}
            \caption{Výsledkem sjednocení mnoha oblastí výzkumu přírodních procesů jsou nové způsoby řešení problémů. Obrázek byl převzat s úpravami z~\cite{FundamentalNatural}.}
            \label{fig:natural-computing-fields}
        \end{figure}
        Příklady výše zmíněných oblastí zahrnují například aplikace umělých neuronových sítí při rozpoznávání hlasu, klasifikaci objektů nebo v oblasti počítačového vidění~\cite{ANN-review,ANN-survey}. 
        Evoluční algoritmy je možné uplatnit při řešení různých optimalizačních problémů, zatímco inteligence roje je aplikovatelná například při návrhu autonomních robotických systémů nebo při prohledávání stavového prostoru~\cite{NaturalComputing,SwarmInteligence}. 
        Umělé imunitní systémy nacházejí uplatnění u komplexních problémů v oblastech od biologie až po robotiku. 
    \item \textbf{Simulace a emulace přírody pomocí výpočetní techniky}
        se zaměřuje na syntézu a studium přírodních fenoménů, známých vzorců či chování a to prostřednictvím simulace, jež je realizována na výpočetních systémech. 
        Tato oblast poskytuje nástroje pro testování biologických teorií, které by bylo obtížné ověřit pomocí experimentálních a analytických metod. 
        Existují dva hlavní přístupy~\cite{FundamentalNatural}:
        \begin{itemize}
            \item Fraktální geometrie v přírodě (\emph{fractal geometry of nature}) poskytuje možnost vizualizovat různé přírodní struktury a procesy, jež se vyznačují nekonečnou úrovní detailu, nekonečnou délkou nebo soběpodobností. 
                Struktury, na nichž je fraktální geometrie viditelná, se nacházejí například na kapradinách, horách nebo brokolici a dokonce je patrná na organismech v podobě fraktální struktury plic, oběhového systému či mozku. 
            \item Umělý život (\emph{artificial life}) je oblast, která se snaží simulovat organismy v umělých prostředích. 
                Hlavním cílem není řešení konkrétního problému, ale pochopení konceptů přírody, včetně vývoje nových forem života. 
        \end{itemize}
        Aplikace fraktální geometrie lze využít při simulaci růstu rostin nebo formování přírodních struktur, zatímco umělý život lze použít ke studiu evoluce a chování organismů ve virtuálním prostředí či k analýze počítačových virů~\cite{FundamentalNatural}. 
    \item \textbf{Výpočty s využitím přírodních materiálů} 
        nepracují s křemíkem jako je tomu u~tradičních výpočetních systémů, ale využívají odlišné materiály, jež umožňují překonat některé limity tradičních počítačů, jelikož poskytují odlišné přístupy k provádění výpočetních operací. 
        Systémy založené na těchto principech nabízejí efektivní řešení problému limit miniaturizace elektroniky založené na křemíku. 
        Alternativní výpočetní prostředky jsou~\cite{FundamentalNatural}:
        \begin{itemize}
            \item Molekulární počítání (\emph{molecular computing}) využívající biologické molekuly, jako je například DNA, k uchování informace, přičemž samotný paralelní výpočet probíhá manipulací s těmito molekulami. 
                Výpočty založené na molekulární úrovni nabízejí vysokou výpočetní rychlost, energetickou efektivitu a levné úložiště informací. 
            \item Kvantové počítání (\emph{quantum computing}) využívající principy kvantové mechaniky, kde je informace uchována na mikroskopické úrovni. 
        \end{itemize}
        Výše uvedené prostředky lze využít k efektivnímu řešení specifických problémů. 
        Molekulární počítání například umožnilo efektivně řešit problém Hamiltonovské cesty~\cite{molecular}, zatímco kvantové počítání se ukázalo jako efektivní například při faktorizaci čísel~\cite{shor}. 
\end{enumerate}
Tato kapitola se dále zaměří na kvantové evoluční počítání, které kombinuje principy kvantového výpočtu a evolučních algoritmů.

\section{Evoluční algoritmy}\label{sec:ea}
Ačkoliv existuje mnoho variant evolučních algoritmů, všechny vycházejí ze stejné myšlenky. 
V prostředí s omezenými zdroji dochází mezi jednotlivci, jejichž schopnosti jsou určeny kvalitativní funkcí, k soutěži o~přežití, jejímž výsledkem je přirozený výběr jedinců, jež během vývoje zvyšují kvalitu celé populace. 
Proces spočívá v aplikaci kvalitativní funkce na náhodně vygenerovanou populaci kandidátních řešení. 
Následně dochází k opakovanému výběru jedinců podstupujících reprodukci aplikací variančních operátorů (typicky křížení a~mutace).
Tento cyklus pokračuje až do nalezení dostatečně kvalitního řešení nebo do dosažení předem nastaveného počtu opakování.  
Evoluční proces, znázorněný v algoritmu~\ref{alg:ea-algo}, lze chápat jako průběžnou optimalizaci, jejímž cílem je nalezení co nejlepšího řešení prostřednictvím postupného přibližování se k~optimálnímu stavu~\cite{IntroductionToEvoComputing}. 

Základní elementy evolučního procesu jsou~\cite{IntroductionToEvoComputing}:
\begin{itemize}
    \item variační operátory (rekombinace a mutace), jenž vytvářejí nové jedince a udržují diverzitu v populaci a
    \item selekce, která napodobuje přirozený výběr preferováním kvalitnějších jedinců. 
\end{itemize}
Evoluční proces je založen na stochastickém chování, přičemž během selekce, rekombinace a mutace figuruje prvek náhody~\cite{IntroductionToEvoComputing}. 

\begin{algorithm}[H]
    \caption{Obecné schéma evolučního algoritmu~\cite{IntroductionToEvoComputing}}
    \label{alg:ea-algo}
    \emph{Inicializace} populace náhodně vygenerovanými kandidátními řešeními\;
    \emph{Ohodnocení} každého kandidátního řešení\;
    \While{\emph{není splněna} ukončovací podmínka}{
        \emph{Selekce} rodičů\;
        \emph{Rekombinace} vybraných rodičů\;
        \emph{Mutace} vzniklých potomků\;
        \emph{Ohodnocení} nových kandidátních řešení\;
        \emph{Selekce} jedinců pro další generaci\;
    }
\end{algorithm}

Podle způsobu reprezentace kandidátních řešení a způsobu realizace některých operátorů lze evoluční algoritmy rozdělit do následujících kategorií~\cite{IntroductionToEvoComputing}:
\begin{itemize}
    \item evoluční strategie (\emph{evolution strategies\,--\,ES}) pracující s reálnými vektory,
    \item evoluční programování (\emph{evolutionary programming\,--\,EP}) využívají stavy automatů,
    \item genetické programování (\emph{genetic programming\,--\,GP}) využívající stromové struktury a
    \item genetické algoritmy (\emph{genetic algorithms\,--\,GA}) reprezentující kandidátní řešení pomocí binárních řetězců. 
\end{itemize}
Rozdíly mezi výše uvedenými kategoriemi jsou především historického charakteru. 
Výběr vhodného typu totiž závisí na povaze řešeného problému, neboť některé typy reprezentací mohou být výhodnější než jiné, například pokud lépe odpovídají charakteru problému, usnadňují reprezentaci kandidátních řešení nebo jsou pro řešený problém přirozenější~\cite{IntroductionToEvoComputing}. 

Pro definici konkrétního evolučního algoritmu je potřeba specifikovat jeho následující části:

\subsubsection*{Reprezentace}
Prvním krokem je definice reprezentace možných řešení problému, která umožňuje efektivní manipulaci s těmito řešeními v rámci algoritmu. 
Možná řešení problému, označovaná jako fenotypy, jsou zakódována do tvarů zvaných jako genotypy, s nimiž algoritmus dokáže pracovat. 
Dochází tak k mapování prostoru fenotypů na prostor genotypů, přičemž ohodnocovací funkce hodnotí fenotypy, ale ohodnocení je přiřazeno genotypům. 
Význam tohoto principu spočívá v tom, že pro algoritmus může být efektivnější pracovat například s~binární reprezentací řešení, i když samotný problém má řešení v oblasti celých čísel, vizte obrázek~\ref{fig:fenotyp-to-genotyp}~\cite{IntroductionToEvoComputing,NaturalComputing}. 
\begin{figure}[ht!]
    \centering
    \includegraphics[width=0.75\textwidth]{FenotypToGenotyp.pdf}
    \caption{Mapování z prostoru genotypů na prostor fenotypů, kde každý fenotyp je následně mapován na hodnotu fitness. Obrázek byl převzat s úpravami z~\cite{NaturalComputing}.}
    \label{fig:fenotyp-to-genotyp}
\end{figure}

V literatuře se v kontextu řešeného problému fenotyp označuje také jako kandidátní řešení nebo jedinec. 
Na straně evolučního algoritmu se pro genotyp často používají označení jako chromozom nebo opět jedinec~\cite{IntroductionToEvoComputing}.

\subsubsection*{Inicializace populace}
V evolučních algoritmech je populace většinou inicializována pomocí náhodně generovaných jedinců. 
Pro inicializaci počáteční populace je možné využít také heuristiku, která, pokud je výpočetně přijatelná, vytvoří kvalitnější populaci~\cite{IntroductionToEvoComputing}. 

\subsubsection*{Ohodnocovací funkce (fitness funkce)}
Hlavním účelem ohodnocovací funkce je řízení evolučního procesu tím, že hodnotí, jak dobře jedinci splňují určené požadavky. 
Jedná se o funkci, jež přiřazuje míru kvality genotypům prostřednictvím ohodnocení odpovídajících fenotypů. 
Ohodnocovací funkce je běžně označována jako fitness funkce. 
V případě optimalizačního problému je často používán pojem objektivní funkce, který vyjadřuje specifické požadavky úlohy (například maximalizaci nebo minimalizaci určité hodnoty).
Ohodnocovací (fitness) funkce $F\left(x\right)$ může být identická s~objektivní funkcí nebo se může jednat o její transformaci:
\begin{equation*}
    F\left( x \right) = g\left( f \left( x \right)\right),
\end{equation*}
kde $x$ je fenotyp, $y=f\left(x\right)$ je objektivní funkce a $g\left(y\right)$ je transformační funkce, jež upravuje hodnotu objektivní funkce pro potřeby selekce a hodnocení v evolučním algoritmu~\cite{IntroductionToEvoComputing,NaturalComputing}. 

\subsubsection*{Populace}
Hlavní rolí populace v evolučních algoritmech je uchování možných řešení problému. 
Populace je chápána jako multimnožina genotypů, která se v průběhu evoluce vyvíjí, přičemž genotypy jsou pouze statické objekty, jenž se samy nemění ani nepřizpůsobují, protože změny nastávají až na úrovni populace, vizte obrázek~\ref{fig:population}. 
\begin{figure}[ht!]
    \centering
    \includegraphics[width=0.67\textwidth]{population.pdf}
    \caption{Populace je multimnožina genotypů, kde se mohou stejné genotypy vyskytovat vícekrát, přičemž složení populace se může měnit s každou novou generací pomocí selekce, rekombinace a mutace.}
    \label{fig:population}
\end{figure}
Velikost populace je obvykle konstantní, což vytváří podmínky pro přirozený výběr jedinců vlivem omezených zdrojů a~nutnosti soutěžení o přežití~\cite{IntroductionToEvoComputing}. 

\subsubsection*{Selekce rodičů}
Rodičem se stává jedinec, jenž byl vybrán k tomu, aby podstoupil proces reprodukce, ať už rekombinací nebo mutací, čímž jsou vytvářeni potomci. 
Obvykle se rodičem stává kvalitní jedinec, avšak vzhledem k tomu, že proces selekce bývá stochastický, může se rodičem stát také méně kvalitní jedinec, což poskytuje vyšší diverzitu, jež pomáhá předcházet uvíznutí v lokálním optimu~\cite{IntroductionToEvoComputing}. 

\subsubsection*{Variační operátory}
Cílem variačních operátorů je vytvořit z existujících jedinců jedince nové. 
V kontextu fenotypového prostoru dochází ke generování nových kandidátních řešení. 
Podle arity se variační operátory dělí na následující dva typy:
\begin{itemize}
    \item \textbf{Mutace} je unární operátor, jenž je aplikován na rodiče a jehož výsledkem je potomek. 
        Tento potomek vzniká na základě série náhodných voleb, která vede k~úpravám genotypu, pokud jsou splněny určité podmínky, vizte obrázek~\ref{fig:mutation}~\cite{IntroductionToEvoComputing,NaturalComputing}. 
        \begin{figure}[ht!]
            \centering
            \includegraphics[width=0.6\textwidth]{Mutation.pdf}
            \caption{Proces mutace rodiče, kde funkce $p\left(x\right)$ určuje pravděpodobnostní rozhodnutí, zda bude provedena mutace konkrétní části genotypu. Na základě této série náhodných rozhodnutí vznikne potomek.}
            \label{fig:mutation}
        \end{figure}
    \item \textbf{Rekombinace}, někdy označovaná jako křížení, je n-ární operátor, jenž kombinuje informace z vícero rodičů. 
        Standardně se v přírodě objevuje pouze binární rekombinace, ale v evolučním počítání je možné uvažovat i křížení s větší aritou. 
        Podobně jako mutace je také rekombinace založena na stochastickém principu, což znamená, že výběr částí, které budou z každého rodiče kombinovány, probíhá na základě náhodné volby. 
        Pro příklad jednoduché jednobodové rekombinace vizte obrázek~\ref{fig:crossover}~\cite{IntroductionToEvoComputing}. 
        \begin{figure}[ht!]
            \centering
            \includegraphics[width=0.8\textwidth]{Crossover.pdf}
            \caption{Ilustrace jednobodové binární rekombinace, kde se na základě náhodně vybraného bodu řezu kombinují informace z rodičovských genotypů do nově vzniklých potomků. Obrázek byl převzat s úpravami z~\cite{NaturalComputing}}
            \label{fig:crossover}
        \end{figure}
\end{itemize}
Mutace v evolučních algoritmech zajišťuje neustálý vývoj evolučního procesu, jelikož v~každé iteraci s určitou pravděpodobností umožňuje objevování nových vlastností. 
Na rozdíl od rekombinace, která, pokud je jediným zdrojem diverzity, přestává vytvářet nová řešení v~případě, že populace konverguje k jednomu genotypu~\cite{NaturalComputing}. 

\subsubsection*{Selekce přeživších}
Podobně jako u selekce rodičů jsou i u selekce přeživších jedinci vybíráni na základě jejich kvality, avšak až poté, co jsou vytvořeni potomci z vybraných rodičů. 
Jelikož velikost populace je ve většině případů konstantní, je nutné rozhodnout, kteří jedinci budou figurovat v~další generaci. 
Toto rozhodnutí může záviset nejen na hodnotě fitness, ale může být zohledněn i věk jedinců, což značí, že selekce jedinců do nové generace bývá ve většině případů deterministická~\cite{IntroductionToEvoComputing}. 

Příkladem jedné z nejběžnějších strategií selekce je sjednocení množiny rodičů a~potomků, jejich seřazení na základě hodnoty fitness a výběr nejlepších jedinců pro novou generaci~\cite{IntroductionToEvoComputing}. 

\subsubsection*{Ukončovací podmínka}
Evoluční proces je ukončen splněním ukončovací podmínky, kterou lze rozdělit do dvou hlavních případů. 
V prvním případě je evoluční algoritmus ukončen po nalezení jedince, jehož hodnota fitness se shoduje s dříve známou optimální hodnotou. 
Ve druhém případě, kdy je řešený problém zjednodušen nebo obsahuje šum, je akceptováno řešení, které se dostatečně blíží optimálnímu řešení v rámci požadované přesnosti. 
Jelikož jsou evoluční algoritmy stochastické, není vždy zaručeno dosažení optimálního řešení, což může vést k~nekončícímu evolučnímu procesu.
Proto je nutné zavést následující podmínky, jejichž splnění zajistí ukončení evoluce~\cite{IntroductionToEvoComputing}:
\begin{enumerate}
    \item Uplyne maximální povolený procesorový čas. 
    \item Je dosaženo maximálního počtu vyhodnocení fitness funkce. 
    \item Po určitou dobu nedochází ke zlepšení fitness nad určitou prahovou hodnotu. 
    \item Diverzita populace klesne pod vybraný práh. 
\end{enumerate}
Pokud není známo optimální řešení problému, postačí pro ukončení evolučního algoritmu použít kteroukoli z výše uvedených podmínek~\cite{IntroductionToEvoComputing}. 

\section{Základy kvantové fyziky}
Tato sekce poskytne stručný popis základních oblastí fyziky a jejich vzájemných souvislostí. 
Následně se zaměří na podrobnější popis vybrané oblasti, konkrétně kvantové mechaniky, jejíž popis je nutný pro hlubší pochopení kapitoly~\ref{chapt:qiea} věnované algoritmům inspirovaným touto oblastí fyziky.

Obrázek~\ref{fig:mechanics} vyobrazuje základní dělení moderní fyziky na elementární části (nepřerušované obdélníky) společně s~jejich vzájemnými vztahy. 
Nepřerušované šipky ukazují nové fyzikální teorie, které vznikly úpravou těch starších a to buď odstraněním některých z jejich předpokladů nebo přidáním nových principů.
Přerušované obdélníky značí nejznámější fyzikálně inspirované algoritmy, přičemž přerušované šipky znázorňují z jakých fyzikálních konceptů tyto algoritmy vycházejí~\cite{NaturalComputing}.

\begin{figure}[ht!]
    \centering
    \includegraphics[width=\textwidth]{mechanics.pdf}
    \caption{Vazby mezi fyzikálními principy a algoritmy. Diagram byl převzat s úpravami z~\cite{NaturalComputing}.}
    \label{fig:mechanics}
\end{figure}

Jednotlivé základní oblasti fyziky společně s jejich stručným popisem jsou následující~\cite{NaturalComputing}:
\begin{itemize}
    \item \textbf{Klasická mechanika (\emph{Classical Mechanics}):} 
    Podle Galileiho a Newtona působí na tělesa síly, které jsou vnímány jako síly působící okamžitě a na dálku. 
    Euler, Laplace, Lagrange, Hamilton a další vytvořili alternativní formulace pomocí skalární energie, které jsou s původní formulací ekvivalentní. 
    \item \textbf{Teorie pole (\emph{Field Theory}):} 
    Koncept formulovaný Maxwellem a dalšími popisující, jak mohou objekty generovat skalární potencionální pole, které vytváří odpovídající vektorové pole (například gravitační nebo elektromagnetické), jež následně ovlivňuje ostatní objekty. 
    Tato pole mohou přenášet energii prostorem ve formě vln, což omezuje rychlost šíření signálů, čímž odstraňuje okamžité působení síly na dálku. 
    Tato teorie mimo jiné ukazuje, že se světlo šíří ve vakuu konstantní konečnou rychlostí. 
    \item \textbf{Speciální teorie relativity (\emph{Special Theory of Relativity}):} 
    Dílo Einsteina, vycházející z toho, že fyzikální zákony jsou ve všech inerciálních vztažných soustavách stejné, společně s faktem, že rychlost světla ve vakuu je vždy konstantní.
    \item \textbf{Obecná teorie relativity (\emph{General Theory of Relativity}):} 
    Rozšířená speciální teorie relativity samotným Einsteinem, která zavádí princip ekvivalence mezi rovnoměrným gravitačním polem a zrychlením. 
    \item \textbf{Kvantová mechanika (\emph{Quantum Mechanics}):} 
    Koncept podle Plancka, Bohra, Schrö\-din\-ge\-ra, Heisenberga, Pauliho, Einsteina a dalších, zavádějící mimo jiné pojem kvantování, který nahrazuje pozorovatelné fyzikální veličiny, jako je například poloha či energie, operátory v Hilbertově prostoru vlnových funkcí.
    \item \textbf{Relativistická kvantová mechanika (\emph{Relativistic Quantum Mechanics}):} 
    Teorie, kde Schrödingerova rovnice pro elektron byla upravena do relativisticky invariantní podoby Diracem tak, že vlnové vektory mají přidané stupně volnosti, které přesně modelují spin elektronu. 
    Tato teorie předpověděla existenci antihmoty, respektive pozitronu. 
    \item \textbf{Kvantová teorie pole (\emph{Quantum Field Theory}):} 
    Dílo Feynmana a dalších, které za pomoci Diracovy práce rozšiřuje relativistickou kvantovou mechaniku tím, že kvantizuje samotné pole včetně vlnové funkce. 
    Tento postup umožňuje interpretovat interakci sil prostřednictvím výměny částic. 
    Praktickým příkladem je Feynmanova kvantová elektrodynamika. 
    \item \textbf{Termodynamika (\emph{Thermodynamics}):} 
    Teorie formulovaná Carnotem, Boltzmannem a dalšími, která se zabývá makroskopickými fyzikálními veličinami, jako je například tlak a teplota, u systémů složených z mikroskopických částic. 
    Popisuje mimo jiné koncept termodynamické rovnováhy, termodynamických procesů a entropie. 
    \item \textbf{Statistická mechanika (\emph{Statistical Mechanics}):} 
    Moderní pohled na termodynamiku, který popisuje makroskopické veličiny pomocí statistického chování částic.
\end{itemize}

Následující sekce se podrobněji zaměří na odnož klasické mechaniky, konkrétně na kvantovou mechaniku, která slouží jako inspirace pro kvantově inspirované algoritmy.

\section{Kvantová mechanika}
Model chování přírodních systémů pozorovaných na velmi krátkých časových a délkových měřítkách popisuje kvantová mechanika, která je rozšířením klasické mechaniky. 
Kvantový systém se může skládat z jedné nebo více částic, jako je například volný elektron či foton~\cite{NaturalComputing}. 

Použitím kvantizace dochází k nahrazení proměnných reprezentujících pozorovatelné fyzikální veličiny, jako jsou poloha, hybnost či energie, lineárními operátory ve vhodném vektorovém prostoru. 
Tento vektorový prostor nese název Hilbertův prostor (\emph{Hilbert space}) a je definován jako úplný unitární prostor nad komplexními čísly. 
Jeho prvky jsou funkce časových a prostorových souřadnic, které reprezentují kvantové stavy systému. 
Kvantový stav systému $\psi$ je někdy rovněž označován jako stavový vektor, vlnová funkce nebo vlnový vektor. 
Na prvky Hilbertova prostoru působí lineární operátory (\emph{observables}), které reprezentují pozorovatelné fyzikální veličiny. 
Vlastní hodnoty těchto operátorů odpovídají možným výsledkům pozorování daných fyzikálních veličin~\cite{NaturalComputing}.

Následující části stručně popisují klíčové koncepty kvantové mechaniky~\cite{NaturalComputing}:

\subsubsection*{Pozorování v kvantové mechanice}
Pozorování kvantového systému vede k určení hodnoty pozorovatelné veličiny, přičemž pravděpodobnost pozorování konkrétní hodnoty odpovídá kvadrátu absolutní hodnoty vlnové funkce $\psi$.
Kvadrát absolutní hodnoty $\left| \psi \right|^2$ představuje hustotu pravděpodobnosti, která udává pravděpodobnost nalezení částice na dané pozici v daném čase v prostoru~\cite{NaturalComputing}. 
    
Časový vývoj vlnové funkce a tedy i hustoty pravděpodobnosti v každém bodě prostoru je popsán lineární Schrödingerovou rovnicí. 
V čase se kvantový stav vyvíjí deterministicky podle této rovnice, což způsobuje rozptyl hustoty pravděpodobnosti v prostoru. 
Kvantový stav se před provedením pozorování nachází v superpozici, tedy v lineární kombinaci všech možných vlastních stavů (\emph{eigenstates}) operátoru odpovídajícího pozorované veličině~\cite{NaturalComputing}. 
    
S postupem času se hustota pravděpodobnosti stále více rozprostírá, což vede ke zvyšování neurčitosti pozice částice. 
Tento proces pokračuje, dokud není provedeno pozorování, při kterém vlnová funkce nelineárně \uv{kolabuje} do jednoho z vlastních stavů měřené veličiny. 
Pravděpodobnost nalezení systému v konkrétním vlastním stavu je dána hodnotou $\left| \psi \right|^2$, přičemž pravděpodobnost nalezení částice je nejvyšší v oblastech, kde je hodnota $\left| \psi \right|$ největší, a nulová tam, kde $\left| \psi \right| = 0$~\cite{NaturalComputing}.
    
Normalizované vlnové funkce splňují podmínku:
\begin{equation*}
    \int_{\mathbb{R}^3} \left|\psi\,(x)\right|^2 dx = 1.    
\end{equation*}
Tato rovnost vyjadřuje, že celková pravděpodobnost nalezení částice v celém prostoru musí být rovna jedné.
Z toho plyne, že kvantové stavy leží na povrchu jednotkové koule Hilbertova prostoru, tedy v množině všech vektorů s normou 1 a operátory, které na ně působí, musí být unitární, aby tuto normu zachovávaly~\cite{NaturalComputing}.

\subsubsection*{Kvantové provázání}
Pojem kvantové provázání (\emph{entanglement}) je jeden z nejdůležitějších rozdílů mezi klasickou a kvantovou fyzikou. 
V kvantovém systému složeného z více částic existuje pouze jedna vlnová funkce, kterou nelze rozdělit na nezávislé vlnové funkce jednotlivých částic. 
Podle Schrödingerovy rovnice se tato vlnová funkce vyvíjí v čase do té doby, dokud není provedeno pozorování.
Kolaps vlnové funkce celého systému nastává při jeho pozorování a okamžitě ovlivňuje stav všech provázaných částic, které následně přecházejí do jednoho z vlastních stavů měřené veličiny~\cite{NaturalComputing}.

V klasickém systému složeném z $m$ částic, kde každá může nabývat $n$ různých stavů, roste dimenze prostoru lineárně jako $n \times m$, neboť celkový prostor lze chápat jako \mbox{$m$-násobný} přímý součet stavových prostorů jednotlivých částic. 
V~kvantovém systému je však správným prostorem tenzorový součin, což způsobuje, že jeho dimenze roste exponenciálně jako $n^m$~\cite{NaturalComputing,QuantumComputing-QuantumInformation}.

\subsubsection*{Dekoherence}
V případě, že dojde ke kontaktu makroskopického prostředí (například měřící zařízení) s~kvantovým systémem, dochází k nevratnému úniku kvantových vlastností do okolí v termodynamicky nevratném procesu. 
Dekoherence se projevuje tím, že složky kombinované vlnové funkce systému a prostředí přestanou efektivně interferovat.
Na makroskopické úrovni tento proces vede k rozpadu superpozic stavů, což se jeví jako zdánlivý kolaps vlnové funkce systému~\cite{NaturalComputing}.

\subsubsection*{Nekomutující operátory}
Pokud jakékoli dva operátory $A$ a $B$ komutují, platí pro ně vztah $\left[A, B \right] = A \times B - B \times A = 0$\footnote{Analogie k Poissonovým závorkám v klasické mechanice.}. 
V kvantové mechanice však operátory obecně nekomutují. 
Příkladem nekomutujících operátorů jsou operátory polohy a hybnosti, jejichž vzájemná nekomutativita vede k Heisenbergově principu neurčitosti.
Tento princip říká, že není možné současně měřit hodnoty obou veličin s požadovanou přesností, respektive měření hodnoty jedné veličiny vede ke snížení přesnosti, s jakou lze určit druhou veličinu~\cite{NaturalComputing,QuantumMeasurement}.

\subsubsection*{Kvantové tunelování}
Kvantové tunelování je jev, při kterém může být částice s~konečnou, nenulovou pravděpodobností pozorována i za bariérou, přestože standardně by ji neměla být schopna překonat. 
Očekávaný čas tunelování závisí nejen na šířce bariéry, ale také na její výšce, přičemž čím širší je bariéra, tím nižší je pravděpodobnost, že částice projde touto bariérou. 
Tento efekt je důsledkem vlnové povahy částice, protože její vlnová funkce je rozprostřena v prostoru, a~tudíž je nenulová i na opačné straně bariéry~\cite{NaturalComputing}. 

\section{Kvantové evoluční počítání}
Hlavní rozdíl mezi kvantovými a klasickými výpočetními systémy spočívá v jejich výpočetních schopnostech.
Kvantové systémy využívají principy superpozice a provázání, což jim umožňuje efektivněji řešit některé problémy, které jsou pro klasické počítače výpočetně náročné. 
Existují dva hlavní přístupy pro kvantové počítače~\cite{NaturalComputing}:
\begin{itemize}
    \item \textbf{Digitální kvantové počítače:} Oproti klasickým počítačům místo bitů využívají tzv. kvantové bity neboli qubity.
    \item \textbf{Adiabatické kvantové počítače:} Hledají optimální řešení problémů prostřednictvím postupné evoluce kvantového systému.
\end{itemize}

Jelikož kvantové výpočty vykazují výhody při řešení určitých typů optimalizačních úloh, vedly k návrhu kvantově inspirovaných evolučních algoritmů, které jsou zařazeny do kvantově evolučního počítání, viz obrázek~\ref{fig:natural-computing}. 
Z tohoto důvodu bude tato sekce věnována právě těmto kvantovým výpočtům~\cite{NaturalComputing}. 
\begin{figure}[ht!]
    \centering
    \includegraphics[width=0.6\textwidth]{NaturalComputing.pdf}
    \caption{Kvantové evoluční počítání je kombinací kvantových a evolučních výpočtů, přičemž obě tyto oblasti spadají do výpočtů inspirovaných přírodou. Obrázek byl převzat s~úpravami z~\cite{QuantumComputing-QuantumInformation}.}
    \label{fig:natural-computing}
\end{figure}

\subsection{Kvantový bit}
Kvantový bit neboli qubit se na rozdíl od klasického bitu, který může nabývat pouze hodnot 0 nebo 1, nachází v superpozici těchto stavů. 
Avšak při měření qubitu dojde ke kolapsu jeho superpozice do jednoho ze stavů klasického bitu~\cite{QuantumComputing-Curious}. 

Superpozice v kontextu klasické fyziky značí situaci, kdy součtem dvou fyzikálních veličin vznikla odlišná fyzikální veličina. 
Případem aplikace této superpozice je výpočet celkové velikosti a směru veličiny, jako je například síla či elektrické pole.
Na rozdíl od klasické fyziky v kvantové mechanice superpozice značí stav, kdy může být systém současně ve více stavech. 
Tento systém zůstává ve stavu superpozice do té doby, dokud není provedeno jeho pozorování, při kterém se superpozice zhroutí a systém bude v jedné konkrétní hodnotě~\cite{QuantumComputing-Curious}. 

Ve standardním zápisu\footnote{Standardně se využívá Diracova nebo \uv{bra-ket} notace.} je kvantový stav qubitu $| \psi \rangle$ vyjádřen jako superpozice stavů $| 0 \rangle$ a $| 1 \rangle$:
\begin{equation}\label{eq:psi=a0+b1}
    | \psi \rangle = \alpha | 0 \rangle + \beta | 1 \rangle, 
\end{equation}
kde koeficienty (amplitudy) $\alpha, \beta \in \mathbb{C}$ umožňují matematicky reprezentovat všechny možné superpozice. 
Tyto koeficienty musí splňovat podmínku normalizace:
\begin{equation}\label{eq:a2+b2=1}
    \left| \alpha \right|^2 + \left| \beta \right|^2 = 1,   
\end{equation}
kde $\left| \alpha \right|^2$, respektive $\left| \beta \right|^2$, udává pravděpodobnost nalezení částice ve stavu $| 0 \rangle$, respektive $| 1 \rangle$, po provedeném měření. 
Podmínka normalizace zajišťuje, že superpozice zkolabuje s~jistotou do jednoho ze stavů $| 0 \rangle$ nebo $| 1 \rangle$~\cite{NaturalComputing,QuantumComputing-Curious}.

Rovnici~\ref{eq:psi=a0+b1} lze díky platnosti rovnosti~\ref{eq:a2+b2=1} přepsat do podoby:
\begin{equation*}
    | \psi \rangle = \cos{\frac{\theta}{2}} | 0 \rangle +  e^{i\phi} \sin{\frac{\theta}{2}} | 1 \rangle,
\end{equation*}
kde hodnoty $\theta$ a $\phi$ určují pozici qubitu na povrchu tzv. Blochovy sféry, viz obrázek~\ref{fig:bloch-sphere}. 
Tato sféra umožňuje vizualizaci stavu pouze jednoho qubitu. 
Pro více qubitů již nelze využít tuto geometrickou reprezentaci~\cite{QuantumComputing-Curious,QuantumComputing-QuantumInformation}. 

\begin{figure}[ht!]
    \centering
    \includegraphics[width=0.41\textwidth]{bloch-sphere.pdf}
    \caption{Blochova sféra reprezentující qubit. Obrázek byl převzat s úpravami z~\cite{QuantumComputing-QuantumInformation}.}
    \label{fig:bloch-sphere}
\end{figure}

\subsection{Kvantová informace}
Princip kvantové provázanosti určuje hlavní rozdíl mezi kvantovým a klasickým bitem, neboť udává silné korelace mezi qubity, což umožňuje pracovat s mnoha stavy současně. 
Tento jev, známý jako kvantový paralelismus, poskytuje kvantovým systémům výpočetní výhodu oproti klasickým systémům, protože umožňuje existenci stavu v podobě superpozice mnoha klasických stavů~\cite{NaturalComputing}.

Vlastnosti kvantové informace způsobující problémy při provádění kvantových výpočtů jsou~\cite{NaturalComputing}:
\begin{itemize}
    \item \textbf{Princip neurčitosti:} V případě nekomutujících pozorovaných veličin ovlivní měření jedné veličiny výsledek měření jiné veličiny. 
    \item \textbf{Princip nemožnosti klonování:} Kvantová informace nemůže být dokonale zkopírována, jinak by byl narušen princip neurčitosti. 
    \item \textbf{Princip provázanosti:} Kvantová informace je díky své provázanosti rozložena mezi více částí systému, což ji typicky znemožňuje rozdělit na části (jednotlivé qubity). Tento princip je hlavním důvodem, proč obecně nelze efektivně simulovat kvantové systémy na těch klasických. 
    \item \textbf{Dekoherence:} Při interakci informací kvantového systému s informacemi z okolního prostředí dochází k jejich vzájemnému provázání, což snižuje schopnost obnovit původní kvantovou informaci, neboť většina provázaných informací pochází z okolního prostředí. 
\end{itemize}

\subsection{Digitální kvantové počítače}
Digitální kvantové počítače pracují s qubity a manipulují s nimi pomocí kvantových hradel (kvantový protějšek logických hradel).
V průběhu výpočtu jsou tyto hradla aplikovány na systém qubitů a měření qubitů je provedeno až na konci samotného výpočtu. 
Paul Benioff uvažoval Turingův stroj pracující s ekvivalentem qubitů a později Richard Feynman dokázal, že klasické počítačové systémy nejsou schopny efektivně simulovat kvantové systémy, zejména kvantové provázání, protože by jejich simulace vyžadovala exponenciální časovou a paměťovou složitost, respektive $\mathcal{O}\left( 2^n \right)$~\cite{NaturalComputing,QuantumComuting-Introduction}. 

Kvantové počítače prokazatelně umožňují oproti těm klasickým řešit některé druhy problémů efektivněji. 
Příklady algoritmů, které jsou schopné řešit problém v kvantovém počítači efektivněji, jsou~\cite{NaturalComputing}: 
\begin{itemize}
    \item \textbf{Shorův algoritmus:} Umožňuje faktorizaci velkých čísel v časové složitosti $\mathcal{O}\left( n^2 \right)$ místo $\mathcal{O}\left(e^{\sqrt[3]{n}} \right)$, jako je tomu u nejlepšího známého algoritmu na klasických systémech.
    \item \textbf{Groverův algoritmus:} Zrychluje vyhledávání v nestrukturovaných databázích klasických systémů z časové složitosti $\mathcal{O}\left( n \right)$ na $\mathcal{O}\left( \sqrt{n} \right)$. 
\end{itemize}
Mimo jiné umožňují kvantové počítače simulovat samy sebe, přičemž však stále nebylo prokázáno zda dokáží řešit NP-těžké problémy v polynomiálním čase\footnote{Obecně se věří, že nedokážou řešit NP-těžké problémy v polynomiálním čase.}~\cite{NaturalComputing}. 

\subsection{Adiabatické kvantové počítače}
Oproti digitálním kvantovým počítačům se adiabatické kvantové počítače neskládají z qubitů. 
Místo toho využívají postupnou evoluci kvantového systému v čase. 
Přístup je podobný analogovým počítačům, kde je vytvořen fyzikální systém odpovídající řešenému problému a jeho stav je ponechán, aby se vyvíjel v čase. 
Jejich výpočetní síla je ekvivalentní výpočetní síle kvantových počítačů založených na qubitech, což znamená, že dokáží řešit stejné problémy se srovnatelnou polynomiální složitostí~\cite{NaturalComputing}.

Adiabatické kvantové počítače splňují kvantový adiabatický teorém, jehož hlavní myšlenkou je, že pokud jsou na systém aplikovány vnější jevy dostatečně pomalu, zůstává ve svém relativním vlastním stavu. 
Pokud jsou však změny vnějších jevů aplikovány na systém příliš rychle, stav systému se nezmění, jelikož nemá dostatek času, aby se přizpůsobil a~skončí v superpozici stavů~\cite{NaturalComputing}. 

Princip evoluce v adiabatických kvantových počítačích je hlavní myšlenkou adiabatického kvantového počítání (\emph{Adiabatic Quantum Computing\,--\,AQC}). 
Tento princip svým charakterem připomíná tepelné žíhání kovů, a proto se rovněž označuje jako kvantové žíhání (\emph{Quantum Annealing\,--\,QA})~\cite{NaturalComputing}. 

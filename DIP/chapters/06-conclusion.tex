\chapter{Závěr}
Tato práce se zabývala metodami návrhu kvantových výpočtů, konkrétně se zaměřovala na kvantově inspirované evoluční algoritmy. 
V úvodu práce byly nastíněny nejdůležitější teoretické základy kvantové mechaniky společně s kvantovým počítáním a posléze byly popsány kvantově inspirované evoluční algoritmy společně s problémem batohu, který byl řešen těmito algoritmy. 
Po vylíčení návrhu experimentů následovalo jejich vyhodnocení, přičemž první experiment naznačil, že kvantově inspirovaný genetický algoritmus dokáže poskytovat kvalitnější řešení než jeho klasická varianta. 
Shodného výsledku bylo dosaženo i u~experimentu zaměřeného na kvantově inspirovaný roj částic, přičemž poslední z experimentů prezentoval porovnání všech testovaných variant algoritmů s takovým závěrem, že kvantově inspirovaný roj částic poskytuje nejkvalitnější a zároveň nejstabilnější výsledky. 

V budoucí práci by mohly být zahrnuty další kvantově inspirované evoluční algoritmy jako je například kvantově inspirované simulované žíhání nebo kvantově inspirovaná diferenciální evoluce. 
Budoucí výzkum se může rovněž věnovat různým přístupům k opravě obsahu batohu po provedeném pozorování, jako je například porovnání náhodného výběru položky s výběrem té, která disponuje nejhorším poměrem váhy a ceny. 

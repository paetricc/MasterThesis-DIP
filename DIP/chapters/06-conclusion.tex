\chapter{Závěr}
Tato diplomová práce se zabývala problematikou optimalizačních algoritmů využívajících principů kvantové fyziky. 
Pro další zpracování byla zvolena skupina kvantově inspirovaných evolučních algoritmů, jež kombinují principy biologické evoluce a kvantové mechaniky. 

V teoretické části byly podrobně popsány jak základy kvantové fyziky, tak hlavní pojmy z oblasti evolučních algoritmů. 
Následně byly představeny čtyři konkrétní kvantově inspirované evoluční algoritmy, jež tyto principy kombinují. 
Konkrétně se jednalo o kvantově inspirovaný genetický algoritmus (\emph{QIGA}), kvantově inspirované simulované žíhání (\emph{QISA}), kvantovou evoluci roje (\emph{QSE}) a kvantově inspirovanou optimalizaci rojem částic (\emph{QIPSO}), přičemž poslední z nich byl navržen v rámci této práce. 

Tyto algoritmy byly experimentálně vyhodnoceny na variantě 0-1 problému batohu, který spadá do skupiny NP-těžkých problémů. 
Nejprve byly na instancích o velikosti 100, 250 a 500 provedeny experimenty za účelem ladění parametrů každého z implementovaných algoritmů. 
Následně byly tyto parametry použity pro testování algoritmů na rozsáhlejších instancích problému o velikosti 1\,000, 2\,000, 5\,000 a 10\,000. 

Výsledky ukázaly, že všechny zkoumané algoritmy jsou schopny produkovat řešení, jež se blíží optimální hodnotě. 
Nejlepšího a zároveň nejstabilnějšího výkonu však dosahoval navržený algoritmus \emph{QIPSO}, který si vedl nejlépe při řešení všech testovaných instancí. 

Z experimentální části vyplývá, že kvantově inspirované evoluční algoritmy představují silnou konkurenci klasickým heuristickým metodám, neboť dosahovaly stabilních výkonů a~zároveň kvalitních výsledků i u rozsáhlých instancí problému. 
Jejich výhodou je vyšší diverzita řešení a schopnost vyhnout se lokálním extrémům díky využití kvantových principů, zejména reprezentace jedinců pomocí pravděpodobnostních koeficientů. 

Do budoucna by bylo možné práci rozšířit o další kvantově inspirované evoluční přístupy, jako je například kvantově inspirovaná diferenciální evoluce nebo mravenčí algoritmy, a to i při řešení jiných typů optimalizačních problémů.

Tato práce byla prezentována na studentské konferenci inovací, technologií a vědy v IT Excel@FIT 2025, kde byla vybrána a oceněna odborným panelem. 

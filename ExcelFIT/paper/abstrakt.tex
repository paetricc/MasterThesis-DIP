%\documentclass{ExcelAtFIT}
\documentclass[czech]{ExcelAtFIT} % when writing in CZECH
%\documentclass[slovak]{ExcelAtFIT} % when writing in SLOVAK

\usepackage[T1]{fontenc}
\usepackage[utf8]{inputenc}

%--------------------------------------------------------
%--------------------------------------------------------
%	REVIEW vs. FINAL VERSION
%--------------------------------------------------------

%   LEAVE this line commented out for the REVIEW VERSIONS
%   UNCOMMENT this line to get the FINAL VERSION
%\ExcelFinalCopy


%--------------------------------------------------------
%--------------------------------------------------------
%	PDF CUSTOMIZATION
%--------------------------------------------------------

\hypersetup{
	pdftitle={Paper Title},
	pdfauthor={Tomáš Bártů},
	pdfkeywords={Keyword1, Keyword2, Keyword3}
}

\lstset{ 
	backgroundcolor=\color{white},   % choose the background color; you must add \usepackage{color} or \usepackage{xcolor}; should come as last argument
	basicstyle=\footnotesize\tt,        % the size of the fonts that are used for the code
}

%--------------------------------------------------------
%--------------------------------------------------------
%	ARTICLE INFORMATION
%--------------------------------------------------------

\ExcelYear{2025}

\PaperTitle{Kvantově inspirované optimalizační algoritmy}

\Authors{Tomáš Bártů*}
\affiliation{*%
  \href{mailto:xbartu11@stud.fit.vutbr.cz}{xbartu11@stud.fit.vutbr.cz},
  \textit{Faculty of Information Technology, Brno University of Technology}}
%%%%--------------------------------------------------------
%%%% in case there are multiple authors, use the following fragment instead
%%%%--------------------------------------------------------
%\Authors{Jindřich Novák*, Janča Dvořáková**}
%\affiliation{*%
%  \href{mailto:xnovak00@stud.fit.vutbr.cz}{xnovak00@stud.fit.vutbr.cz},
%  \textit{Faculty of Information Technology, Brno University of Technology}}
%\affiliation{**%
%  \href{mailto:xdvora00@stud.fit.vutbr.cz}{xdvora00@stud.fit.vutbr.cz},
%  \textit{Faculty of Information Technology, Brno University of Technology}}


%--------------------------------------------------------
%--------------------------------------------------------
%	ABSTRACT and TEASER
%--------------------------------------------------------

\Abstract{
Tato práce se zaměřuje na srovnání vybraných kvantově inspirovaných evolučních algoritmů využívajících principů kvantové fyziky v prostředí klasického výpočetního systému. 
Konkrétně byla testována kvantově inspirovaná varianta genetického algoritmu, částicového systému a simulovaného žíhání, přičemž řešenou úlohou byl problém batohu ve variantě 0/1. 
V rámci práce bylo dále navrženo a~implementováno vylepšení kvantově inspirovaného algoritmu částicového systému. 
Experimenty ukázaly, že nově navržený přístup dosahuje lepších výsledků než jeho původní varianta a překonává i ostatní testované kvantově inspirované evoluční algoritmy a to i při řešení větších instancí problému. 
Výsledky experimentů rovněž naznačují, že kvantově inspirované přístupy obecně dosahují lepších výsledků než jejich klasické varianty. 
}

%--------------------------------------------------------
%--------------------------------------------------------
%--------------------------------------------------------
%--------------------------------------------------------
\begin{document}

\startdocument

%--------------------------------------------------------
%--------------------------------------------------------
%	ARTICLE CONTENTS
%--------------------------------------------------------

%%%%%%%%%%%%%%%%%%%%%%%%%%%%%%%%%%%%%%%%%%%%%%%%%%%%%%%%%%%%
%%%%%%%%%%%%%%%%%%%%%%%%%%%%%%%%%%%%%%%%%%%%%%%%%%%%%%%%%%%%
%%%%%%%%%%%%%%%%%%%%%%%%%%%%%%%%%%%%%%%%%%%%%%%%%%%%%%%%%%%%
%%%%%%%%%%%%%%%%%%%%%%%%%%%%%%%%%%%%%%%%%%%%%%%%%%%%%%%%%%%%
\section{Úvod}
\label{sec:Introduction}

\textbf{[Motivace]}
Optimalizační algoritmy inspirované přírodními procesy se běžně využívají při řešení různých typů složitých úloh, jenž nejsou snadno řešitelné klasickými metodami. 
Existují přístupy, které do těchto heuristik začleňují principy \textbf{kvantové mechaniky}, což slibuje efektivnější prohledávání prostoru řešení.

\textbf{[Definice problému]}
Cílem práce je vytvořit srovnávací studii kvantově inspirovaných evolučních algoritmů. 
Srovnání proběhlo na \textbf{problému batohu} ve variantě 0/1, který patří mezi NP-úplné kombinatorické problémy. 
Součástí srovnávací studie je také porovnání kvantově inspirovaných evolučních algoritmů s jejich klasickými variantami, přičemž u~všech algoritmů byly předem laděny jejich parametry. 

\textbf{[Existující řešení]}
Mezi používané klasické heuristické algoritmy patří genetický algoritmus, částicové systémy a simulované žíhání. 
Pro každý z těchto přístupů existují i kvantově inspirované varianty a to kvantově inspirovaný genetický algoritmus (\emph{QIGA})~\cite{qiga}, kvantově inspirovaná evoluce roje (\emph{QSE})~\cite{qse} a kvantově inspirované simulované žíhání (\emph{QISA})~\cite{qisa}.

\textbf{[Naše řešení]}
V této práci byly testovány schopnosti kvantově inspirovaných evolučních algoritmů při řešení různých instancí problému batohu. 
Hodnocení algoritmů se zaměřovalo nejen na kvalitu nalezeného řešení, ale také na jeho škálovatelnost a schopnost efektivně řešit různé velikosti instancí problému. 
Součástí práce je rovněž návrh, implementace a experimentální ověření vlastního kvantově inspirovaného algoritmu \emph{QIPSO}, který vznikl vhodnou úpravou algoritmu \emph{QSE}.

%%%%%%%%%%%%%%%%%%%%%%%%%%%%%%%%%%%%%%%%%%%%%%%%%%%%%%%%%%%%
%%%%%%%%%%%%%%%%%%%%%%%%%%%%%%%%%%%%%%%%%%%%%%%%%%%%%%%%%%%%
%%%%%%%%%%%%%%%%%%%%%%%%%%%%%%%%%%%%%%%%%%%%%%%%%%%%%%%%%%%%
%%%%%%%%%%%%%%%%%%%%%%%%%%%%%%%%%%%%%%%%%%%%%%%%%%%%%%%%%%%%
\section{Kvantově inspirované evoluční algoritmy}
\label{sec:qiea}

Kvantově inspirované evoluční algoritmy (\emph{QIEA}) jsou heuristické metody, jenž využívají vybrané principy kvantové fyziky, zejména superpozice stavů a~pravděpodobnostní reprezentaci. 
Stav jedince v populaci je reprezentován kvantovým bitem (\emph{qubit}), jenž je vyjádřen lineární kombinací stavů $| 0 \rangle$ a $| 1 \rangle$:
\begin{equation*}\label{eq:psi=a0+b1}
	| \psi \rangle = \alpha | 0 \rangle + \beta | 1 \rangle, 
\end{equation*}
kde ${\left| \alpha \right|}^2 + {\left| \beta \right|}^2 = 1$, přičemž $\alpha$ a $\beta$ reprezentují amplitudy pravděpodobnosti výběru jednotlivých stavů. 

% Kvantová populace čítající $m$ jedinců je reprezentována pravděpodobnostní maticí: 
% \begin{equation}\label{eq:quantum-representation}
%     \begin{bmatrix}
%         \alpha_1 & \alpha_2 & \dots & \alpha_m \\
%         \beta_1  & \beta_2  & \dots & \beta_m
%     \end{bmatrix},
% \end{equation}
% kde $\alpha_i^2 + \beta_i^2 = 1$ v případě normalizovaného systému.

Přestože \emph{QIEA} využívají principy kvantové fyziky, nej\-sou cíleny pro běh na kvantovém počítači, ale jsou určeny k běhu na klasických výpočetních systémech. 
Pro ověření schopností těchto algoritmů byl použit problém batohu ve variantě 0/1, kde každá položka může být buď vybrána, nebo nevybrána pro vložení do batohu, vizte \fbox{(1)}.

Všechny zkoumané \emph{QIEA} využívají kvantovou reprezentaci řešení~\fbox{(2)}, kde pravděpodobnostní koeficienty $\alpha_i$ a $\beta_i$ $i$-tého qubitu jsou aktualizovány pomocí kvantového rotačního hradla, vizte~\fbox{Obrázek 1} a rovnici~\fbox{(3)}.
Konkrétní velikost změny úhlu $\Delta\theta_i$ je určena na základě aktuálně pozorovaného binárního řešení $x_i$ a nejlepšího známého řešení $b_i$ podle tabulky~\fbox{Tabulka 1}.

Všechny zkoumané algoritmy využívají kvantovou reprezentaci řešení a jejich aktualizace je prováděna prostřednictvím kvantového rotačního hradla.
V~této práci bylo konkrétně experimentováno se čtyřmi typy \emph{QIEA}:
\begin{itemize}
	\item \textbf{QIGA\,--\,} kvantové rozšíření klasického genetického algoritmu.
	\item \textbf{QISA\,--\,} úprava simulovaného žíhání, jež je doplněna o metodu tepelně-řízeného pozorování (\emph{heat observation}).
	\item \textbf{QSE\,--\,} kvantově inspirovaná verze částicového systému, kde je poloha částic ovlivněna jak osobním, tak globálním nejlepším řešením.
	\item \textbf{QIPSO\,--\,} námi navržená modifikace algoritmu QSE, která zohledňuje také aktuálně pozorované binární řešení při aktualizaci stavů.
\end{itemize}
Testované algoritmy byly rovněž porovnávány s jejich klasickými variantami.

%%%%%%%%%%%%%%%%%%%%%%%%%%%%%%%%%%%%%%%%%%%%%%%%%%%%%%%%%%%%
%%%%%%%%%%%%%%%%%%%%%%%%%%%%%%%%%%%%%%%%%%%%%%%%%%%%%%%%%%%%
%%%%%%%%%%%%%%%%%%%%%%%%%%%%%%%%%%%%%%%%%%%%%%%%%%%%%%%%%%%%
%%%%%%%%%%%%%%%%%%%%%%%%%%%%%%%%%%%%%%%%%%%%%%%%%%%%%%%%%%%%
\section{Návrh algoritmu QIPSO}
Algoritmus \emph{QIPSO}, jenž je znázorněn pomocí pseudokódu v \fbox{Algoritmu 1}, pracuje s kvantovou reprezentací řešení popsanou maticí~\fbox{(2)}.
Hlavní rozdíl oproti původnímu \emph{QSE} spočívá v tom, jak probíhá aktualizace stavu částic. 
Zatímco \emph{QSE} upravuje kvantové úhly na základě nejlepšího osobního a globálního řešení, \emph{QIPSO} navíc využívá \textbf{aktuálně pozorovaná binární řešení}. 

Rychlost částic (\emph{velocity}) v \emph{QIPSO} je ovlivněna nejen nejlepším globálním a osobním řešením, ale také tím, jak moc se aktuální pozorované řešení liší od těchto nejlepších řešeních, čímž algoritmus lépe vyjadřuje aktuální rozložení populace. 
Dále byl zaveden parametr tření (\emph{friction}), který pomáhá tlumit příliš prudké změny rychlosti a tím zajišťuje stabilní vývoj populace. 

%%%%%%%%%%%%%%%%%%%%%%%%%%%%%%%%%%%%%%%%%%%%%%%%%%%%%%%%%%%%
%%%%%%%%%%%%%%%%%%%%%%%%%%%%%%%%%%%%%%%%%%%%%%%%%%%%%%%%%%%%
%%%%%%%%%%%%%%%%%%%%%%%%%%%%%%%%%%%%%%%%%%%%%%%%%%%%%%%%%%%%
%%%%%%%%%%%%%%%%%%%%%%%%%%%%%%%%%%%%%%%%%%%%%%%%%%%%%%%%%%%%
\section{Srovnávací studie}
Cílem experimentální části je ověřit schopnosti algoritmů \emph{QIEA} při řešení úlohy batohu a vytvořit srovnávací studii. 
Součástí experimentů je rovněž ladění parametrů jednotlivých algoritmů, jenž probíhalo s~instancemi $100$, $250$ a $500$ položek batohu. 
Následně byly spuštěny experimenty s již doladěnými parametry na větších instancích batohu s $1000$, $2000$ a $5000$ položkami. 
Každé experimentální nastavení algoritmu a jeho vyhodnocení bylo podloženo třiceti nezávislými běhy (\fbox{Tabulka 2}).

Již při ladění parametrů jednotlivých algoritmů se ukázalo, že \emph{QIEA} dosahují obecně kvalitnějších výsledků než jejich klasické varianty. 
Při experimentování s~většími instancemi problému batohu (1000 položek) bylo patrné, že \emph{QIEA} se dokáží výrazně přiblížit optimálnímu řešení, přičemž námi navržený algoritmus \emph{QIPSO} poskytoval ze všech testovaných přístupů nejkvalitnější řešení~(\fbox{Graf 1}). 
Při zvětšení instance problému batohu na 5000 položek sice všechny testované algoritmy zaznamenaly pokles přesnosti, avšak \emph{QIPSO} i v tomto případě dosahoval kvalitnějších řešení než ostatní \emph{QIEA} (\fbox{Graf 2}). 

%%%%%%%%%%%%%%%%%%%%%%%%%%%%%%%%%%%%%%%%%%%%%%%%%%%%%%%%%%%%
%%%%%%%%%%%%%%%%%%%%%%%%%%%%%%%%%%%%%%%%%%%%%%%%%%%%%%%%%%%%
%%%%%%%%%%%%%%%%%%%%%%%%%%%%%%%%%%%%%%%%%%%%%%%%%%%%%%%%%%%%
%%%%%%%%%%%%%%%%%%%%%%%%%%%%%%%%%%%%%%%%%%%%%%%%%%%%%%%%%%%%
\section{Závěr}
Tato práce se zabývala vytvořením srovnávací studie kvantově inspirovaných evolučních algoritmů při řešení kombinatorického problému batohu ve variantě 0/1. 
V rámci práce byl navržen a implementován vlastní algoritmus \emph{QIPSO}, jenž rozšiřuje přístup algoritmu \emph{QSE}.
Experimenty naznačují, že kvantově inspirovaný přístup u evolučních algoritmů dosahuje lepších výsledků než klasický přístup, přičemž algoritmus \emph{QIPSO} vykazuje nejlepší výsledky ze všech testovaných metod. 

%%%%%%%%%%%%%%%%%%%%%%%%%%%%%%%%%%%%%%%%%%%%%%%%%%%%%%%%%%%%
%%%%%%%%%%%%%%%%%%%%%%%%%%%%%%%%%%%%%%%%%%%%%%%%%%%%%%%%%%%%
%%%%%%%%%%%%%%%%%%%%%%%%%%%%%%%%%%%%%%%%%%%%%%%%%%%%%%%%%%%%
%%%%%%%%%%%%%%%%%%%%%%%%%%%%%%%%%%%%%%%%%%%%%%%%%%%%%%%%%%%%
\section*{Poděkování}
Rád bych poděkoval svému vedoucímu diplomové práce doc. Ing. Michalu Bidlovi Ph.D. za jeho ochotu, cenné rady a odbornou pomoc, kterou mi v průběhu práce poskytl.

%--------------------------------------------------------
%--------------------------------------------------------
%--------------------------------------------------------
%	REFERENCE LIST
%--------------------------------------------------------
%--------------------------------------------------------
\phantomsection
\bibliographystyle{unsrt}
\bibliography{bibliography}

%--------------------------------------------------------
%--------------------------------------------------------
%--------------------------------------------------------
\end{document}